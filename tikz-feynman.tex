%%%%%%%%%%%%%%%%%%%%%%%%%%%%%%%%%%%%%%%%%%%%%%%%%%%%%%%%%%%%%%%%%%%%%%%%%%%%%%%%
%
% Feynman Diagrams with TikZ
% Copyright (C) 2014  Joshua Ellis
%
% Allows Feynman diagrams to be used with TikZ.
%
%
%% This LaTeX file is free: you can redistribute it and/or modify it under the
% terms of the GNU General Public License as published by the Free Software
% Foundation, either version 3 of the License, or (at your option) any later
% version.
%
% This is distributed in the hope that it will be useful, but WITHOUT ANY
% WARRANTY; without even the implied warranty of MERCHANTABILITY or FITNESS FOR
% A PARTICULAR PURPOSE.  See the GNU General Public License for more details.
%
%%%%%%%%%%%%%%%%%%%%%%%%%%%%%%%%%%%%%%%%%%%%%%%%%%%%%%%%%%%%%%%%%%%%%%%%%%%%%%%%

%%%%%%%%%%%%%%%%%%%%%%%%%%%%%%%%%%%%%%%%%%%%%%%%%%%%%%%%%%%%%%%%%%%%%%%%%%%%%%%%
%% HEADER
%%%%%%%%%%%%%%%%%%%%%%%%%%%%%%%%%%%%%%%%%%%%%%%%%%%%%%%%%%%%%%%%%%%%%%%%%%%%%%%%

\def\pgfautoxrefs{1}
\documentclass[a4paper,final]{ltxdoc}

%% Formatting
%%%%%%%%%%%%%%%%%%%%%%%%%%%%%%%%%%%%%%%%%%%%%%%%%%%%%%%%%%%%%%%%%%%%%%%%%%%%%%%%

\usepackage{multicol}
\usepackage[
    a4paper,
    hmargin=2.25cm,
    vmargin=2.5cm,
    nohead
]{geometry} % Easily change margin sizes (and many more dimensions)

%\usepackage{setspace} % Line spacing
%\singlespacing        % 1-spacing (default)
%\onehalfspacing       % 1,5-spacing
%\doublespacing        % 2-spacing

%% Language
%%%%%%%%%%%%%%%%%%%%%%%%%%%%%%%%%%%%%%%%%%%%%%%%%%%%%%%%%%%%%%%%%%%%%%%%%%%%%%%%

\usepackage[UKenglish]{babel}
\usepackage[T1]{fontenc}
% \usepackage{microtype}
\usepackage{pifont}

%% Graphics & Figure
%%%%%%%%%%%%%%%%%%%%%%%%%%%%%%%%%%%%%%%%%%%%%%%%%%%%%%%%%%%%%%%%%%%%%%%%%%%%%%%%

\usepackage[pdftex]{graphicx}   % For loading graphic files

\usepackage{tikz}
\usepackage{tikz-feynman}

% Set up PGF externalization
% Note that this requires the folder pgf-img to already exist
\usetikzlibrary{external}
\tikzexternalize[shell escape=-shell-escape, prefix=pgf-img/]
\tikzset{
    external/mode=list and make,
    external/system call={
        lualatex \tikzexternalcheckshellescape -halt-on-error -interaction=batchmode -jobname="\image" "\texsource" || rm "\image.pdf"},
}


%% Math Packages
%%%%%%%%%%%%%%%%%%%%%%%%%%%%%%%%%%%%%%%%%%%%%%%%%%%%%%%%%%%%%%%%%%%%%%%%%%%%%%%%

\usepackage{amsmath,amsfonts,amssymb} % The core packages for math
\usepackage{mathtools}                % Extra features

% Add tags to referenced lines only
\mathtoolsset{showonlyrefs,showmanualtags}
% Define \withnumber which forces a single line to have a number
\def\withnumber{\refstepcounter{equation}\tag{\theequation}}

% Allows page breaks in math (1 = avoid if possible, 4 = whenever)
% Page breaks can be avoided at particular places by using \\*
\allowdisplaybreaks[2]

%% Other Packages
%%%%%%%%%%%%%%%%%%%%%%%%%%%%%%%%%%%%%%%%%%%%%%%%%%%%%%%%%%%%%%%%%%%%%%%%%%%%%%%%

\usepackage{enumitem}    % Can customize {enumerate} and {itemize} lists
\usepackage{hyperref}    % Automatically inserts hyperlinks.
\usepackage{listings}    % Code listings
\usepackage{fp}          % Floating point arithmetics
\usepackage{minted}      % Use Pygments
\usepackage{makeidx}
\usepackage{xr}          % Cross-referencing

%% Other modifications
%%%%%%%%%%%%%%%%%%%%%%%%%%%%%%%%%%%%%%%%%%%%%%%%%%%%%%%%%%%%%%%%%%%%%%%%%%%%%%%%

% Modify the skip after each paragraph
\setlength{\parskip}{1ex plus 0.5ex minus 0.2ex}
% \setlength{\headheight}{25.23pt}

\definecolor{link-color}{RGB}{0 0 75}
\definecolor{cite-color}{RGB}{75 0 25}
\definecolor{file-color}{RGB}{75 25 0}
\definecolor{url-color}{RGB}{75 25 0}
\definecolor{link-border-color}{RGB}{100 200 255}
\definecolor{cite-border-color}{RGB}{255 0 150}
\definecolor{url-border-color}{RGB}{255 150 0}

\hypersetup{
    pdftitle={tikz-feynman: Feynman diagrams with TikZ},
    pdfkeywords={Feynman diagrams; TeX; LaTeX; ConTeXt; Tikz; pgf; tikz-feynman},
    colorlinks=true,      % If colorlinks is false, a border is drawn instead
                          % which does not appear in print.  If it is true, the
                          % font is coloured and does appear in print.
    linkcolor=link-color,
    citecolor=cite-color,
    filecolor=file-color,
    urlcolor=url-color,
    linkbordercolor=link-border-color,
    citebordercolor=cite-border-color,
    urlbordercolor=url-border-color,
}

\providecommand{\LuaTeX}{Lua\TeX}

\IfFileExists{pgfmanual-en-macros}
  {\input{pgfmanual-en-macros}}
  {\PackageError{tikz-feynman-manual}{
This document requires the file pgfmanual-en-macros.tex (distributed
with pgf) to compile.  Please place a copy of that file in the current
directory}{}}

\makeindex

\pgfkeys{
  /pdflinks/search key prefixes in=
    {/tikz/feynman/,/tikz/graphs/feynman/}  % This really needs to be commented out!
  /pdflinks/internal link prefix=tikzfeynman,
	%
  /pdflinks/warnings=false,
	% for debugging:
  /pdflinks/show labels=false,
}

%%%%%%%%%%%%%%%%%%%%%%%%%%%%%%%%%%%%%%%%%%%%%%%%%%%%%%%%%%%%%%%%%%%%%%%%%%%%%%%%
%%     DOCUMENT
%%%%%%%%%%%%%%%%%%%%%%%%%%%%%%%%%%%%%%%%%%%%%%%%%%%%%%%%%%%%%%%%%%%%%%%%%%%%%%%%
\begin{document}

\begin{center}
\vspace*{1em}
\tikz\node[scale=1.2]{%
  \color{gray}\Huge\ttfamily \char`\{\textcolor{red!75!black}{tikz-feynman}\char`\}};

\vspace{0.5em}
{\Large\bfseries Feynman diagrams with \tikzname}

\vspace{0.7em}
{Version 0.1.2 \qquad \today}

\vspace{1.3em}
{by  Joshua Ellis}
\end{center}

\vfill

\begin{center}
\begin{codeexample}[graphic=white]
\begin{tikzpicture}
  \graph [feynman, node distance=2.5cm, edges={thick}, vertical= e to f]
  {
    a -- [fermion] b -- [photon] c -- [fermion] d,
    b -- [fermion, momentum=\(p_{1}\)] e -- [fermion, momentum=\(p_{2}\)] c,
    e -- [gluon]  f,
    h -- [fermion] f -- [fermion] i;
  };
\end{tikzpicture}\end{codeexample}
\end{center}

\vfill

\begin{multicols}{2}
  \tableofcontents
\end{multicols}
%%%%%%%%%%%%%%%%%%%%%%%%%%%%%%%%%%%%%%%%%%%%%%%%%%%%%%%%%%%%%%%%%%%%%%%%%%%%%%%%
%% CONTENT
%%%%%%%%%%%%%%%%%%%%%%%%%%%%%%%%%%%%%%%%%%%%%%%%%%%%%%%%%%%%%%%%%%%%%%%%%%%%%%%%

\newpage
\section{Introduction}
\label{sec:introduction}

This package provides a set of pre-defined styles in order to draw Feynman
diagrams using \href{https://www.ctan.org/pkg/pgf}{\tikzname} more easily and
consistently.  The set of styles defined here were originally inspired by
\href{http://tex.stackexchange.com/a/87395/26980}{this answer} on
\href{http://tex.stackexchange.com}{|http://tex.stackexchange.com|}, so due
credit must go to Jake.

If you have any suggestions or have found any bugs, please feel free to create a
new issue or pull request on the Github page:
\href{https://www.github.com/JP-Ellis/tikz-feynman}{|https://www.github.com/JP-Ellis/tikz-feynman|}.

\subsection{Installation}
\label{subsec:installation}

This package is \emph{not} currently offered on
\href{https://www.ctan.org}{CTAN} as it is just a personal project of mine;
however, if enough people find it useful, I will look into making it available
through CTAN.

In order to use this as it is, simply download |tikz-feynman.sty| and place it
in the same directory as your \TeX~file and include it using the usual
|\usepackage{tikz-feynman}|.  Alternatively, it is also possible to install
|tikz-feynman| system-wide by placing it inside \TeX's search path (which will
vary based on your operating system).

In v3.0.0 of \tikzname, there is a bug in the Lua component of the graphdrawing
library which prevents it from handling coordinate nodes properly.  This bug
does not seem to affect the usual \tikzname~drawing library.  If you wish to use the
|\graph| command with any of the options that require Lua, you will need to
apply the following patch:

\inputminted[fontsize=\footnotesize, bgcolor=codebackground]{diff}{pgf.patch}


\newpage
\section{Usage}
\label{sec:usage}

|tikz-feynman| has three ways of setting up the Feynman diagram.  The placement
of vertices can either be fully-automated using some algorithm; specified
related to other vertices; or fully manual using coordinates.  Each method is
mostly compatible with the others, so it is possible to specify a an initial
set of vertices using one of the graph algorithms, and then place additional
vertices relative to these.

There is one exception: a |\graph| with |feynman spring layout| or |feynman
 electrical layout| must consist entirely of new nodes and \emph{cannot} anchor
to nodes defined outside the graph.

The three methods of placing nodes are illustrated below and see also the
examples for uses in different contexts.


\subsection{Automatic Placement}
\label{subsec:automatic_placement}

The \tikzname~graphdrawing library offers the ability to automatically position the
vertices of a Feynman diagram by following an algorithm.  For some of these
algorithm, \LuaTeX~is required as the edges are modelled by springs, and the
vertices may be given charges.

|tikz-feynman| pre-defines three graph styles: |feynman spring layout|, |feynman
 electrical layout| and |feynman layered layout|.  By default, when using
|\graph [feynman]|, the spring layout is used which models each edge as springs.

\begin{codeexample}[]
\tikz \graph [feynman, horizontal'=a to b] {
    a1 -- [fermion, edge label=\(e^{-}\)] a [label=70:\(g\)] -- [fermion, edge label=\(e^{+}\)] a2,
    a  -- [photon, edge label=\(\gamma\)] b,
    b1 -- [fermion, edge label=\(e^{+}\)] b -- [fermion, edge label=\(e^{-}\)] b2;
  };\end{codeexample}


\subsection{Semi-automatic Placement}
\label{subsec:semi-automatic_placement}

\tikzname~also provides the ability to place vertices relative to other previously
labelled vertices using various |above=of name|, |left=of name|, and similar
keys.  |tikz-feynman| also provides the command |\vertex| which just a shorthand
for |\node[vertex]|.  In the future, |\vertex| is intended to intelligently
recognize when a vertex has a name and adapt the style to display the name.

Once the nodes have been placed, it is possible to use a simple |\graph|
environment in order to draw in the edges, or alternatively, using the |\draw| command.

\begin{codeexample}[]
\begin{tikzpicture}[feynman]
  \vertex [label=70:\(g\)] (a) {};
  \vertex (b) [right=of a] {};
  \vertex (a1) [above left=of a] {};
  \vertex (a2) [below left=of a] {};
  \vertex (b1) [above right=of b] {};
  \vertex (b2) [below right=of b] {};

  \graph {
    (a1) -- [fermion, edge label=\(e^{-}\)] (a) [label=70:\(g\)] -- [fermion, edge label=\(e^{+}\)] (a2),
    (a)  -- [photon, edge label=\(\gamma\)] (b);
    (b1) -- [fermion, edge label'=\(e^{+}\)] (b) -- [fermion, edge label'=\(e^{-}\)] (b2);
  };
\end{tikzpicture}\end{codeexample}


\subsection{Manual Placement}
\label{subsec:manual_placement}

Lastly, it is possible to fully specify each vertex' coordinates.

\begin{codeexample}[]
\begin{tikzpicture}[feynman]
  \vertex[label=70:\(g\)] (a) at (-1, 0) {};
  \vertex (b)  at (1, 0)       {};
  \vertex (a1) at (-2.5, 1.5)  {};
  \vertex (a2) at (-2.5, -1.5) {};
  \vertex (b1) at (2.5, 1.5)   {};
  \vertex (b2) at (2.5, -1.5)  {};

  \graph {
    (a1) -- [fermion, edge label=\(e^{-}\)] (a) [label=70:\(g\)] -- [fermion, edge label=\(e^{+}\)] (a2),
    (a)  -- [photon, edge label=\(\gamma\)] (b);
    (b1) -- [fermion, edge label'=\(e^{+}\)] (b) -- [fermion, edge label'=\(e^{-}\)] (b2);
  };
\end{tikzpicture}\end{codeexample}


\clearpage
\section{Examples}
\label{sec:examples}

\begin{center}
\begin{codeexample}[]
\begin{tikzpicture}
  \graph [feynman, horizontal=a to b1]
  {
    ai [particle=\(e^{-}\)] -- [fermion] a -- [fermion] af [particle=\(e^{+}\)],
    a  -- [photon] b1 -- [fermion, semi-left] b2 -- [fermion, semi-left] b1,
    b2 -- [photon] c,
    ci [particle=\(e^{+}\)] -- [fermion] c -- [fermion] cf [particle=\(e^{-}\)];
  };
\end{tikzpicture}\end{codeexample}

\begin{codeexample}[]
\begin{tikzpicture}
  \graph [feynman, horizontal=b1 to b3]
  {
    ai -- [fermion] a -- [fermion] af,
    a  -- [photon] b1,
    b3 -- [photon] c,
    ci -- [fermion] c -- [fermion] cf;
    {[edges={fermion, looseness=1}]
      b1
      -- [out=90, in=180] b2
      -- [out=0, in=90] b3
      -- [out=-90, in=0] b4
      -- [out=180, in=-90] b1,
    };
  };
  \draw[gluon] (b2) -- (b4);
\end{tikzpicture}\end{codeexample}

\begin{codeexample}[]
\begin{tikzpicture}
  \graph [feynman, vertical=e to f]
  {
    a -- [fermion] b -- [photon] c -- [fermion] d,
    b -- [fermion] e -- [fermion] c,
    e -- [gluon]  f,
    h -- [fermion] f -- [fermion] i;
  };
\end{tikzpicture}\end{codeexample}

\begin{codeexample}[]
\begin{tikzpicture}[feynman]
  \graph [feynman layered layout, grow=right, edges={fermion}] {
    a -- b -- c -- d -- e
  };
  \vertex (v) [below=of c] {};
  \vertex (h) [below=of v] {};

  \draw[charged scalar] (b) to [out=-90, in=180] (v);
  \draw[charged scalar] (v) to [out=0, in=-90] (d);
  \draw[scalar] (v) to (h);
\end{tikzpicture}\end{codeexample}

\begin{codeexample}[]
\begin{tikzpicture}
  \graph [feynman electrical layout, horizontal=a to b] {
    { [edges={charged scalar}]
      a -- b -- c -- d -- a
    },
    a1 --[fermion] a,
    b1 --[anti fermion] b,
    c1 --[fermion] c,
    d1 --[anti fermion] d;
  };
\end{tikzpicture}\end{codeexample}

\begin{codeexample}[]
\begin{tikzpicture}[feynman]
  \vertex (a1) {};
  \vertex (a2) [right=of a1] {};
  \vertex (a3) [right=4cm of a2] {};
  \vertex (b2) [above=1cm of a3] {};
  \vertex (b1) [left=2.5cm of b2] {};
  \vertex (c)  [above=0.5cm of b2] {};
  \vertex (s1) [below=0.5cm of a1] {};
  \vertex (s2) [below=0.5cm of a3] {};

  \graph {
    {[edges={fermion}]
      (a1) -- (a2) -- (a3),
      (b1) -- (b2),
      (b1) -- (c),
      (s1) -- (s2),
    },
    (a2) -- [scalar] (b1),
  };
\end{tikzpicture}\end{codeexample}

\begin{codeexample}[]
\begin{tikzpicture}[feynman]
  \vertex (a1) {};
  \vertex (a2) [right=of a1] {};
  \vertex (a3) [right=4cm of a2] {};
  \vertex (b2) [below=0.5cm of a3] {};
  \vertex (b1) [below left=0.75cm and 3cm of b2] {};
  \vertex (b3) [below=1.5cm of b2] {};
  \vertex (s1) [below=2.5cm of a1] {};
  \vertex (s2) [below=2.5cm of a3] {};

  \graph {
    {[edges={fermion}]
      (a1) -- (a2) -- (a3),
      (b1) -- (b2),
      (b1) -- (b3),
      (s1) -- (s2),
    },
    (a2) -- [scalar] (b1),
  };
\end{tikzpicture}\end{codeexample}

\begin{codeexample}[]
\begin{tikzpicture}[feynman]
  \vertex (a1) {};
  \vertex (a2) [below=4cm of a1] {};
  \vertex (b1) [below right=1cm and 2cm of a1] {};
  \vertex (b2) [above right=1cm and 2cm of a2] {};
  \vertex (c1) [right=5cm of a1] {};
  \vertex (c2) [right=5cm of a2] {};

  \graph {
    { [edges=fermion]
      (a1) -- (b1),
      (c2) -- (b1),
      (b2) -- (a2),
      (b2) -- (c1),
    },
    (b1) -- [photon] (b2),
  };
\end{tikzpicture}\end{codeexample}
\end{center}


\clearpage
\section{Documentation}
\label{sec:documentation}

\begin{key}{/tikz/feynman}
  Sets certain options within the scope to be so that they work consistently
  across the various positioning methods.  Note that any |\graph|

  Sets the |below=of name| spacing to values consistent with the way graphs will
  place the nodes.
\end{key}


\subsection{Graph Drawing}
\label{subsec:graph_drawing}

The following keys are defined for the |\graph| command.  Please refer to the
graphdrawing documentation in the main \tikzname~manual for additional information.

\begin{key}{/tikz/graphs/feynman}
  The default style for Feynman diagrams; simply a shorthand for |feynman spring layout|.
\end{key}

\begin{key}{/tikz/graphs/every feynman}
  Provides the basic underlying style to all Feynman diagrams created using
  |\graph|.

\begin{codeexample}[]
\tikzset{graphs/every feynman/.append style={edges={red, thick}}}
% ...
\tikz \graph [feynman, horizontal=c to d] {
  {a, b} -- c -- [photon] d -- {e, f}
};
\end{codeexample}
\end{key}

\begin{key}{/tikz/graphs/feynman spring layout}
  Models each edge as a spring when determining the final placement of the
  vertices.  This requires \LuaTeX.

\begin{codeexample}[]
\tikz \graph [feynman spring layout, horizontal=c to d] {
  {a, b} -- c -- [photon] d -- {e, f}
};
\end{codeexample}
\end{key}

\begin{key}{/tikz/graphs/feynman electrical layout}
  Models each edge as a spring and gives each vertex a charge when determining
  the final placement of the vertices.  This requires \LuaTeX.

\begin{codeexample}[]
\tikz \graph [feynman electrical layout, horizontal=c to d] {
  {a, b} -- c -- [photon] d -- {e, f}
};
\end{codeexample}
\end{key}

\begin{key}{/tikz/graphs/feynman layered layout}
  Models each edge as a spring and gives each vertex a charge when determining
  the final placement of the vertices.  This requires \LuaTeX.

\begin{codeexample}[]
\tikz \graph [feynman layered layout, grow=right] {
  {a, b} -- c -- [photon] d -- {e, f}
};
\end{codeexample}
\end{key}


\subsection{Edge Styles}
\label{subsec:edge_styles}

\begin{keylist}{/tikz/with arrow,/tikz/with reversed arrow}
  Adds an arrow in the middle pointing forwards in the case of |with arrow|, or
  pointing backward in the case of |with reversed arrow|.

\begin{codeexample}[]
\begin{tikzpicture}
  \draw[with arrow]          (0, 1) to (2, 1);
  \draw[with reversed arrow] (0, 0) to (2, 0);
\end{tikzpicture} \end{codeexample}
\end{keylist}

\begin{key}{/tikz/photon}
  Sinusoidal line for photons.

\begin{codeexample}[]
\tikz \draw[photon] (0, 0) to (2, 0);
\end{codeexample}
\end{key}

\begin{key}{/tikz/scalar}
  Dashed line for scalars.

\begin{codeexample}[]
\tikz \draw[scalar] (0, 0) to (2, 0);
\end{codeexample}
\end{key}

\begin{keylist}{/tikz/charged scalar,/tikz/anti charged scalar}
  Dashed line with an arrow for charged scalars.  The arrow is reversed for 
  |anti charged scalar|.

\begin{codeexample}[]
\begin{tikzpicture}
  \draw[charged scalar]      (0, 1) to (2, 1);
  \draw[anti charged scalar] (0, 0) to (2, 0);
\end{tikzpicture} \end{codeexample}
\end{keylist}

\begin{keylist}{/tikz/fermion,/tikz/anti fermion}
  Solid line with an arrow for fermions.  The arrow is reversed for 
  |anti fermion|.

\begin{codeexample}[]
\begin{tikzpicture}
  \draw[fermion]      (0, 1) to (2, 1);
  \draw[anti fermion] (0, 0) to (2, 0);
\end{tikzpicture} \end{codeexample}
\end{keylist}

\begin{key}{/tikz/gluon}
  Coils for gluons.

\begin{codeexample}[]
\tikz \draw[gluon] (0, 0) to (2, 0);
\end{codeexample}
\end{key}


\subsubsection{Momentum Arrows}
\label{subsubsec:momentum_arrows}

\begin{keylist}{
    /tikz/momentum=\meta{label} (default empty),
    /tikz/momentum'=\meta{label} (default empty)}
  Draw an arrow parallel to the edge with \meta{label} if specified.  The
  alternative |momentum'| places the arrow on the other side of the edge.

  The separation between the edge and the arrow, and the label and the arrow can
  be changed through the |momentum/distance| and |momentum/label distance|
  keys.  Similarly, the distance by which the arrows are shortened compared to
  the edge is specified in |momentum/shorten|.

\begin{codeexample}[]
\begin{tikzpicture}
  \draw[momentum'=\(p_1\)]  (0, 0.5) to (2, 0.5);
  \draw[momentum=\(p_2\)]   (0, 0) to (2, 0);
\end{tikzpicture} \end{codeexample}
\end{keylist}

\begin{keylist}{
    /tikz/reversed momentum=\meta{label} (default empty),
    /tikz/reversed momentum'=\meta{label} (default empty),
    /tikz/rmomentum=\meta{label} (default empty),
    /tikz/rmomentum'=\meta{label} (default empty)}
  The same as |momentum| and |momentum'| respectively, with the arrow direction
  reversed.  The |rmomentum| and |rmomentum'| keys are simply abbreviations.

\begin{codeexample}[]
\begin{tikzpicture}
  \draw[reversed momentum'=\(p_1\)]  (0, 0.5) to (2, 0.5);
  \draw[reversed momentum=\(p_2\)]   (0, 0) to (2, 0);
\end{tikzpicture} \end{codeexample}
\end{keylist}

\begin{key}{/tikz/momentum/distance=\meta{distance} (default 3mm)}
  Specify the distance separating the arrow and edge

\begin{codeexample}[]
\begin{tikzpicture}
  \draw[momentum/distance=5mm, momentum] (0, 1) to (2, 1);
  \draw[momentum/distance=1mm, momentum] (0, 0) to (2, 0);
\end{tikzpicture} \end{codeexample}
\end{key}

\begin{key}{/tikz/momentum/shorten=\meta{distance} (default 4mm)}
  Specify the amount by which the momentum arrows are shortened compared to the
  end.

\begin{codeexample}[]
\begin{tikzpicture}
  \draw[momentum/shorten=4mm, momentum] (0, 1) to (2, 1);
  \draw[momentum/shorten=1mm, momentum] (0, 0) to (2, 0);
\end{tikzpicture} \end{codeexample}
\end{key}

\begin{key}{/tikz/momentum/label distance=\meta{distance} (default 2.5mm)}
  Specify the distance separating the momentum arrow label and the momentum
  arrow.

\begin{codeexample}[]
\begin{tikzpicture}
  \draw[momentum/label distance=3mm, momentum=\(p_1\)] (0, 1) to (2, 1);
  \draw[momentum/label distance=1mm, momentum=\(p_1\)] (0, 0) to (2, 0);
\end{tikzpicture} \end{codeexample}
\end{key}



\subsubsection{Edge Modifiers}
\label{subsubsec:edge_modifiers}

\begin{key}{/tikz/semi-left}
  Causes the edge to turn left and complete a semicircle until it reaches the
  next node.

\begin{codeexample}[]
\tikz \graph[horizontal=a to b] { a --[semi-left] b --[semi-left] a };
\end{codeexample}
\end{key}

\begin{key}{/tikz/semi-right}
  Same as |/tikz/semi-left|, but going around the other way.
\end{key}


\subsection{Vertex Styles}
\label{subsec:vertex_styles}

\begin{key}{/tikz/vertex}
  The base node style used in Feynman diagram.
\end{key}

\begin{key}{/tikz/every vertex}
  A style applied to all vertices in a Feynman diagram.

\begin{codeexample}[]
\tikzset{every vertex/.style={red, shape=circle}}
% ...
\tikz \graph[feynman, horizontal=a to b] {a -- b};
\end{codeexample}
\end{key}

\begin{key}{/tikz/particle=\meta{name}}
  Place the particle \meta{name} at the location of the vertex.  This should
  only be used for terminal vertices.

\begin{codeexample}[]
\tikz \graph[feynman, horizontal=a to b] {
  a [particle=\(e^{-}\)] -- [fermion] b [particle=\(e^{-}\)]
};
\end{codeexample}
\end{key}

\begin{key}{/tikz/dot}
  Style the vertex as a dot.

\begin{codeexample}[]
\tikz \graph[feynman layered layout, grow=right] {
  {a, b} -- c [dot] -- [photon] d [dot] -- { e, f }
};
\end{codeexample}
\end{key}

\begin{key}{/tikz/blob}
  Style the vertex as a blob.

\begin{codeexample}[]
\tikz \graph[feynman layered layout, grow=right] {
  {a, b} -- c [blob] -- {d, e, f}
};
\end{codeexample}
\end{key}

\printindex

\end{document}

%%% Local Variables:
%%% mode: latex
%%% TeX-master: t
%%% End:
