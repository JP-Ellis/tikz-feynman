%%%%%%%%%%%%%%%%%%%%%%%%%%%%%%%%%%%%%%%%%%%%%%%%%%%%%%%%%%%%%%%%%%%%%%%%%%%%%%%%
%
% TikZ-Feynman
% Feynman Diagrams with TikZ
% Copyright (C) 2016  Joshua Ellis
%
%
% This LaTeX file is free: you can redistribute it and/or modify it under the
% terms of the GNU General Public License as published by the Free Software
% Foundation, either version 3 of the License, or (at your option) any later
% version.
%
% This is distributed in the hope that it will be useful, but WITHOUT ANY
% WARRANTY; without even the implied warranty of MERCHANTABILITY or FITNESS FOR
% A PARTICULAR PURPOSE.  See the GNU General Public License for more details.
%
%%%%%%%%%%%%%%%%%%%%%%%%%%%%%%%%%%%%%%%%%%%%%%%%%%%%%%%%%%%%%%%%%%%%%%%%%%%%%%%%

%%%%%%%%%%%%%%%%%%%%%%%%%%%%%%%%%%%%%%%%%%%%%%%%%%%%%%%%%%%%%%%%%%%%%%%%%%%%%%%%
%% HEADER
%%%%%%%%%%%%%%%%%%%%%%%%%%%%%%%%%%%%%%%%%%%%%%%%%%%%%%%%%%%%%%%%%%%%%%%%%%%%%%%%

\def\pgfautoxrefs{1}
\documentclass[a4paper,final]{ltxdoc}

\usepackage{etex}

%% Formatting
%%%%%%%%%%%%%%%%%%%%%%%%%%%%%%%%%%%%%%%%%%%%%%%%%%%%%%%%%%%%%%%%%%%%%%%%%%%%%%%%

\usepackage[vmargin=4cm, hmargin=3cm]{geometry}  % Customize text width, page height, margins, etc.
% \usepackage{multicol}  % {multicols}{n} environment
% \usepackage{pdflscape} % {landscape} environment

%\usepackage{setspace} % Line spacing
%\singlespacing        % 1-spacing (default)
%\onehalfspacing       % 1,5-spacing
%\doublespacing        % 2-spacing

\usepackage{fontspec}  % Allows other fonts to be specified
\setmainfont{EB Garamond}[
  Contextuals={Alternate},
  Numbers=OldStyle,
]
\setmonofont{Inconsolata-g}[
  Scale=MatchLowercase,
]
\newfontface\swshape{EB Garamond 12 Italic}[Style=Swash]

\usepackage{microtype}   % Fine small typographical details
\usepackage{titlesec}    % Change section heading font
\titleformat*{\section}{\huge\scshape}
\titleformat*{\subsection}{\Large\scshape}
\titleformat*{\subsubsection}{\large\scshape}
\titleformat*{\paragraph}{\scshape}

\usepackage[perpage]{footmisc}
\usepackage{enumitem}    % Easily customize lists
\setlist[description]{font=\normalfont\scshape}

%% Change the Table of Contents fonts
\usepackage{tocloft}
\renewcommand{\cftsecfont}{\normalfont\scshape}
\setlength\cftbeforesecskip{0pt}
\setcounter{tocdepth}{2}

%% Language
%%%%%%%%%%%%%%%%%%%%%%%%%%%%%%%%%%%%%%%%%%%%%%%%%%%%%%%%%%%%%%%%%%%%%%%%%%%%%%%%

\usepackage[UKenglish]{babel} % Set up the language
\usepackage{csquotes}

%% Graphics & Figure
%%%%%%%%%%%%%%%%%%%%%%%%%%%%%%%%%%%%%%%%%%%%%%%%%%%%%%%%%%%%%%%%%%%%%%%%%%%%%%%%

\usepackage{graphicx}   % Allow graphics to be included
\usepackage{xcolor}     % Define and use colours
% \usepackage{subcaption} % Subfigures inside a figure

% Keep all pictures in the './img/' sub-directory.
\graphicspath{{./img/}}

\usepackage{tikz}       % Powerful drawing language
\usepackage{tikz-feynman}
\makeatletter
\tikzfeynmanset{compat=\tikzfeynman@version@major.\tikzfeynman@version@minor.\tikzfeynman@version@patch}
\makeatother

%% TikZ pictures and plots can significantly increase the time it takes to
%% produce the output.  The `external` TikZ library library defers the creation
%% of these figures to a sub-process which creates a separate PDF file which is
%% then simply imported into the main document.  To call the sub-process, you
%% have to execute the appropriate makefile.  If you are using LatexMk, you can
%% use the `.latexmkrc` to automatically do this for you.
%%
%% The following setup works on Linux, and should work on OS X too.
% \usetikzlibrary{external}
% \tikzexternalize[shell escape=-shell-escape, prefix=pgf-img/]
% \immediate\write18{mkdir -p pgf-img}
% \tikzset{
%     external/mode=list and make,
%     external/system call={
%         lualatex \tikzexternalcheckshellescape -halt-on-error -interaction=batchmode -jobname="\image" "\texsource" || rm "\image.pdf"},
% }

%% Math Packages
%%%%%%%%%%%%%%%%%%%%%%%%%%%%%%%%%%%%%%%%%%%%%%%%%%%%%%%%%%%%%%%%%%%%%%%%%%%%%%%%

\usepackage{amsmath}   % The core math package
\usepackage{amssymb}   % Defines additional math fonts
\usepackage{mathtools} % Various extra maths functions
% \usepackage{cancel}    % Show cancellations with \cancel{}

\usepackage[cmintegrals,varg]{newtxmath} % Nice math with Garamond

%% Define \withnumber which forces the line to have number
\newcommand{\withnumber}{\refstepcounter{equation}\tag{\theequation}}

%% Allows page breaks in math (1 = avoid if possible, 4 = whenever)
%% Page breaks can be avoided at particular places by using \\*
\allowdisplaybreaks[2]

\mathtoolsset{
  %% Number only referenced equations
  showonlyrefs,
}


\DeclarePairedDelimiter{\angles}{\langle}{\rangle}

%% Other Packages
%%%%%%%%%%%%%%%%%%%%%%%%%%%%%%%%%%%%%%%%%%%%%%%%%%%%%%%%%%%%%%%%%%%%%%%%%%%%%%%%

\usepackage{listings}    % Code listings
\usepackage{makeidx}     % Index creation
\usepackage[orig,UKenglish]{isodate}     % Handle dates

\usepackage{hyperref}    % Automatically inserts hyperlinks.
\usepackage{cleveref}    % Use `\cref{}` to reference anything

%% Bibliography
%%%%%%%%%%%%%%%%%%%%%%%%%%%%%%%%%%%%%%%%%%%%%%%%%%%%%%%%%%%%%%%%%%%%%%%%%%%%%%%%

%% hyperref should be loaded first
\usepackage[
    backend=biber,
    style=phys,
    autocite=inline,
    biblabel=brackets,
    eprint=true,
    doi=true,
    isbn=true,
]{biblatex}

\addbibresource{references.bib}

%% Other modifications
%%%%%%%%%%%%%%%%%%%%%%%%%%%%%%%%%%%%%%%%%%%%%%%%%%%%%%%%%%%%%%%%%%%%%%%%%%%%%%%%

% Modify the skip after each paragraph
\setlength{\parskip}{1ex plus 0.5ex minus 0.2ex}
\setlength{\parindent}{0pt}

\providecommand{\LuaTeX}{Lua\TeX}
\providecommand{\tikzfeynmanname}{\tikzname-Feynman}
\providecommand{\pgfmanual}{\href{http://mirrors.ctan.org/graphics/pgf/base/doc/pgfmanual.pdf}{\tikzname{} manual}}

\IfFileExists{pgfmanual-en-macros}{
  \let\attribute\relax
  \input{pgfmanual-en-macros}
}{
  \PackageError{tikz-feynman-manual}{
This document requires the file pgfmanual-en-macros.tex (distributed
with pgf) to compile.  Please place a copy of that file in the current
directory}{}}

\makeindex

\pgfkeys{
  /pdflinks/search key prefixes in={/tikzfeynman/}
  /pdflinks/internal link prefix=tikzfeynman,
  %
  /pdflinks/warnings=false,
  % for debugging:
  /pdflinks/show labels=false,
}

\makeatletter
%% Make table of content heading smaller
\renewcommand\tableofcontents{%
    \section*{\normalfont\large\scshape\contentsname}%
    \vspace{-1.5\baselineskip}%
    \@starttoc{toc}%
}
\makeatother

%% Define some slightly nicer colors
\definecolor{link-color}{RGB}{96 0 0}
\definecolor{cite-color}{RGB}{0 96 0}
\definecolor{file-color}{RGB}{0 0 96}
\definecolor{url-color}{RGB}{0 0 96}
\definecolor{link-border-color}{RGB}{255 159 159}
\definecolor{cite-border-color}{RGB}{159 255 159}
\definecolor{file-border-color}{RGB}{159 159 255}
\definecolor{url-border-color}{RGB}{159 159 255}

\hypersetup{
  pdfborder=0 0 0,
}

%% Change the abstract style
\renewenvironment{abstract}{
  \begin{center}
    \begin{minipage}{0.9\textwidth}
      \begin{center}
        \Large
        Abstract
      \end{center}\vspace*{-1em}
      \rule{\textwidth}{1pt}
}{
      \par\noindent
      \rule{\textwidth}{1pt}
    \end{minipage}
  \end{center}
}

%% Put the abstract only on the arXiv version
\newif\ifarxiv
\arxivfalse

%% Acknowledgements go in two different places depending on whether it is the
%% arXiv version or not.  Have the common code here.
\def\acknowledgements{
  \subsection*{Acknowledgements}
  \label{subsec:acknowledgements}

  The original proof-of-concept for using \tikzname{} to draw Feynman diagrams
  was done by the user `\href{http://tex.stackexchange.com/users/2552}{Jake}' on
  the \TeX{} StackExchange.  His original answer can be viewed at:
  \url{http://tex.stackexchange.com/a/87395/26980}.

  I must also thank all the people who have used the development versions of
  \tikzfeynmanname{} and offered suggestions to improve it and found bugs for me
  to fix.

  This work was partially supported by the N.~D.~Goldsworthy Scholarship for
  Physics.
}

%% LaTeX Documentation
\usepackage{doc}
\providecommand\marg[1]{%
  {\ttfamily\char`\{}\meta{#1}{\ttfamily\char`\}}}
\providecommand\oarg[1]{%
  {\ttfamily[}\meta{#1}{\ttfamily]}}
\providecommand\parg[1]{%
  {\ttfamily(}\meta{#1}{\ttfamily)}}


%% Center codeexample
\makeatletter
\let\codeexample@orig=\codeexample
\let\endcodeexample@orig=\endcodeexample
\def\codeexample{
  \begin{center}
    \codeexample@orig
}
\def\endcodeexample{
    \endcodeexample@orig
  \end{center}
}

%% Document Information
%%%%%%%%%%%%%%%%%%%%%%%%%%%%%%%%%%%%%%%%%%%%%%%%%%%%%%%%%%%%%%%%%%%%%%%%%%%%%%%%

\makeatletter
\newcommand{\@keywords}{}
\newcommand{\keywords}[1]{\renewcommand{\@keywords}{#1}}
\newcommand{\@institution}{}
\newcommand{\institution}[1]{\renewcommand{\@institution}{#1}}

\AtBeginDocument{
  \hypersetup{
    pdftitle={\@title},
    pdfauthor={\@author},
    pdfkeywords={\@keywords},
  }
}
\makeatother

\title{\tikzfeynmanname}
\author{Joshua P.~\textsc{Ellis}}
\institution{
  ARC Centre of Excellence for Particle Physics at the Terascale \\
  School of Physics, The University of Melbourne \textsc{vic} 3010, Australia
}
\keywords{Feynman diagram, TeX, LaTeX, TikZ, pgf, TikZ-Feynman}

%%%%%%%%%%%%%%%%%%%%%%%%%%%%%%%%%%%%%%%%%%%%%%%%%%%%%%%%%%%%%%%%%%%%%%%%%%%%%%%%
%% DOCUMENT
%%%%%%%%%%%%%%%%%%%%%%%%%%%%%%%%%%%%%%%%%%%%%%%%%%%%%%%%%%%%%%%%%%%%%%%%%%%%%%%%
\begin{document}
\pagenumbering{roman}

\begin{titlepage}
  \makeatletter
  \begin{center}
    \vspace*{1em}
    \tikz\node[scale=1.5]{%
      \color{gray}\Huge\ttfamily \char`\{\textcolor{red!75!black}{\@title}\char`\}};


    \vspace{0.5em}
    {\huge Feynman diagrams with \tikzname}

    \vspace{0.7em}
    {Version \texttt{\tikzfeynman@version} \qquad \origdate\printdateTeX{\tikzfeynman@date}}

    \vspace{1.3em}
    {by \@author} \\[1em]
    {\@institution}
  \end{center}
  \makeatother

  \vfill

\begin{codeexample}[graphic=white]
\feynmandiagram [large, vertical=e to f] {
  a -- [fermion] b -- [photon, momentum=\(k\)] c -- [fermion] d,
  b -- [fermion, momentum'=\(p_{1}\)] e -- [fermion, momentum'=\(p_{2}\)] c,
  e -- [gluon]  f,
  h -- [fermion] f -- [fermion] i;
};
\end{codeexample}

  \vfill

  \begin{multicols}{2}
    \tableofcontents
  \end{multicols}
\end{titlepage}

%%%%%%%%%%%%%%%%%%%%%%%%%%%%%%%%%%%%%%%%%%%%%%%%%%%%%%%%%%%%%%%%%%%%%%%%%%%%%%%%
%% CONTENT
%%%%%%%%%%%%%%%%%%%%%%%%%%%%%%%%%%%%%%%%%%%%%%%%%%%%%%%%%%%%%%%%%%%%%%%%%%%%%%%%
\clearpage
\pagenumbering{arabic}

\ifarxiv
\begin{abstract}
  \tikzfeynmanname{} is a \LaTeX{} package allowing Feynman diagrams to be
  easily generated within \LaTeX{} with minimal user instructions and without
  the need of external programs.  It builds upon the \tikzname{} package and
  leverages the graph placement algorithms from \tikzname{} in order to automate
  the placement of many vertices.  \tikzfeynmanname{} still allows fine-tuned
  placement of vertices so that even complex diagrams can be generated with
  ease.
\end{abstract}
\fi

\section{Introduction}
\label{sec:introduction}

\tikzfeynmanname{} provides a new way to draw Feynman diagrams in \LaTeX{} that
does not rely on external programs and uses a clear extensible syntax.

%% Trust people on arXiv to cite TikZ-Feynman, but otherwise put a notice to
%% remind people
\ifarxiv\else
If you use \tikzfeynmanname{} in an academic setting, please cite:
\begin{quote}
  \fullcite{tikz-feynman}
\end{quote}
\fi

Feynman diagrams provide a description of interactions of subatomic particles in
a form that is clearer and more succinct than the corresponding mathematical
description.  They were introduced by \citeauthor{PhysRev.76.769} and first
appear in his paper \citetitle{PhysRev.76.769} \cite{PhysRev.76.769}.  Since
then, \LaTeX{} has become widely used to type-set scientific papers and
currently, two leading methods of drawing Feynman diagrams in \LaTeX{} are
\href{https://www.ctan.org/pkg/feynmf}{|feynMF|/|feynMP|} \cite{feynmf} and
\href{https://www.ctan.org/pkg/axodraw}{|AxoDraw|} \cite{axodrawV1,axodrawV2},
with the latter also featuring a \textsc{gui} front-end called
\href{http://jaxodraw.sourceforge.net}{|JaxoDraw|} \cite{jaxodrawV1,jaxodrawV2}.

In |feynMF|/|feynMP|, the position of vertices follows an algorithm whilst
|AxoDraw| requires their position to be specified on a grid.  As an example, the
code to generate an \(\ell\ell \to \ell\ell\) scattering Feynman diagram in each
package is:
\begin{center}
  \begin{tabular}{p{0.3\linewidth} p{0.27\linewidth} p{0.34\linewidth}}
    |AxoDraw| & |feynMF|/|feynMP| & \tikzfeynmanname{} \\[-2em]
\begin{codeexample}[execute code=false]
\begin{picture}
\ArrowLine(190,270)(160,300)
\ArrowLine(160,240)(190,270)
\ArrowLine(270,300)(240,270)
\ArrowLine(240,270)(270,240)
\Photon(190,270)(240,270){4}{4.5}
\Vertex(190,270){1.5}
\Vertex(240,270){1.5}
\end{picture}
\end{codeexample}
              &
\begin{codeexample}[execute code=false]
\begin{fmfchar*}(40,30)
\fmfpen{thick}
\fmfleft{i1,i2}
\fmfright{o1,o2}
\fmf{fermion}{i1,v1,o1}
\fmf{fermion}{i2,v2,o2}
\fmf{photon,label=$q$}{v1,v2}
\fmfdot{v1,v2}
\end{fmfchar*}
\end{codeexample}
                                  &
\begin{codeexample}[execute code=false]
\feynmandiagram [horizontal=a to b] {
  i1 -- [fermion] a -- [fermion] i2,
  a -- [photon] b,
  f1 -- [fermion] b -- [fermion] f2,
};
\end{codeexample}
  \end{tabular}
\end{center}

The inconvenience of specifying absolute coordinates in |AxoDraw| is alleviated
by |JaxoDraw|, though this requires an external program.  In addition, the
creation of Feynman diagrams in |feynMF|/|feynMP| and |AxoDraw| requires further
processing outside of \LaTeX{}.\footnote{With the exception of |AxoDraw| when
  compiled with |latex|.}

\tikzfeynmanname{} on the other hand requires no external program and delegates
the positioning of vertices to algorithms originally developed by
\citeauthor{hu2005} \cite{hu2005}, and \citeauthor{eades1991} \cite{eades1991},
both of which were implemented into \tikzname{} by \citeauthor{pohlmann2011}
\cite{pohlmann2011}.  Since \tikzfeynmanname{} is built on \tikzname{}, users
can harness the power and extensibilty of \tikzname{} \cite{tikz} making it easy
to extend to accommodate individual needs.  In order to produce more complicated
diagrams, relative or absolute positioning of vertices can also be used in
\tikzfeynmanname{} so that any diagram can be generated with relative ease.

\tikzfeynmanname{} is made available through the
\href{https://ctan.org/pkg/tikz-feynman}{Comprehensive \TeX{} Archive Network
  (\textsc{ctan})}\footnote{\url{https://ctan.org/pkg/tikz-feynman}} and can
also be downloaded from the
\href{http://www.jpellis.me/projects/tikz-feynman}{project
  page}\footnote{\url{http://www.jpellis.me/projects/tikz-feynman}}.  The
project is open source and contributions are welcome.  The management of bugs
and feature requests is done at
\href{https://github.com/JP-Ellis/tikz-feynman}{Github}\footnote{\url{https://github.com/JP-Ellis/tikz-feynman}}.

\tikzfeynmanname{}'s versioning will approximately follow
\href{http://semver.org}{semantic versioning}.  This means that changes in the
third number (|1.0.0| to |1.0.1|) will consist of bug fixes and very minor
changes but they should not change the output otherwise\footnote{That is, with
  the exception of the bug that they are fixing.}.  Changes in the second number
(|1.0.0| to |1.1.0|) will consist of new features and everything should be
backwards compatible.  Finally, changes in the first number (|1.0.0| to |2.0.0|)
indicates a major change in the package and code written for |1.0.0| is not
guaranteed to work on |2.0.0|.  The intended version of this package to use
should be indicated when the package is loaded with
|\usepackage[compat=x.y.z]{tikz-feynman}| so the user may be informed of any
discrepancy.  If needed, earlier versions may be downloaded from the
\href{http://www.jpellis.me/projects/tikz-feynman}{project
  page}\footnote{\url{http://www.jpellis.me/projects/tikz-feynman}}.

\subsection*{Licence}
\label{subsec:licence}

The \emph{documentation} may be redistributed and/or modified under the terms of
the \textsc{Gnu} General Public License as published by the Free Software
Foundation, either version 3 of the License, or (at your option) any later
version.

The \emph{code of this package} may be distributed and/or modified under the
terms of the \textsc{Gnu} General Public License as published by the Free
Software Foundation, either version 3 of the License, or (at your option) any
later version.

This package is distributed in the hope that it will be useful, but without any
warranty; without even the implied warranty of merchantability or fitness for a
particular purpose.

\ifarxiv\else
\acknowledgements
\fi


\clearpage
\section{Tutorial}
\label{sec:tutorial}

\subsection{Loading the Package}
\label{sibsec:loading_the_package}

After installing the package, the \tikzfeynmanname{} package can be loaded with
|\usepackage{tikz-feynman}| in the preamble of the \TeX{} document.  It is
recommend that you also specify the version of \tikzfeynmanname{} to use with
the optional package argument |compat|:
|\usepackage[compat=1.1.0]{tikz-feynman}|.  This ensures that any new versions
of \tikzfeynmanname{} do not produce any undesirable changes without warning.

\subsection{A First Diagram}
\label{subsec:a_first_diagram}

Feynman diagrams can be declared with the |\feynmandiagram| command.  It is
analogous to the |\tikz| command from \tikzname~and requires a final semi-colon
(|;|) to finish the environment.  For example, a simple \(s\)-channel diagram
is:

\begin{codeexample}[]
\feynmandiagram [horizontal=a to b] {
  i1 -- [fermion] a -- [fermion] i2,
  a -- [photon] b,
  f1 -- [fermion] b -- [fermion] f2,
};
\end{codeexample}
\begin{description}
\item[Line 1] |\feynmandiagram| introduces the Feynman diagram and allows for
  optional arguments to be given in the brackets \texttt{[\meta{options}]}.  In
  this instance, |horizontal=a to b| orients the algorithm outputs such that the
  line through vertices |a| and |b| is horizontal.
\item[Line 2] The left fermion line is drawn by declaring three vertices (|i1|,
  |a| and |i2|) and connecting them with edges |--|.  Just like the
  |\feynmandiagram| command above, each edge also takes optional arguments
  specified in brackets \texttt{[\meta{options}]}.  In this instance, these
  edges need arrows in order to indicate that they are fermion lines, so the
  |fermion| style is added to them.

  Optional arguments can also be given to the vertices in exactly the same way.
\item[Line 3] This edge connects vertices |a| and |b| with an edge styled as a
  photon.  Since there is already a vertex labelled |a|, the algorithm will
  connect it to a new vertex labeled |b|.
\item[Line 4] This line is analogous to line 2 and introduces two new vertices,
  |f1| and |f2|.  It re-uses the previously labelled |b| vertex.
\item[Line 5] Finish the declaration of the Feynman diagram.  The final
  semi-colon (|;|) is important.
\end{description}

The name given to each vertex in the graph does not matter.  So in this example,
|i1|, |i2| denote the initial particles; |f1|, |f2| denotes the final particles;
and |a|, |b| are the end points of the propagator; however, any other labels
could have been given.  The only important aspect is that what was called |a| in
line 2 is also |a| in line 3 so that the underlying algorithm treats them as the
same vertex.

The order in which vertices are declared does not matter as the default
algorithm re-arranges everything\footnote{It is possible for the algorithm to
  get a little confused in some circumstances, but these cases should be rather
  rare.  For some algorithms (such as the |layered~layout|), the order in which
  vertices are introduced \emph{does} matter.  This is documented in
  \cref{subsubsec:diagram_keys}.}.  For example, one might prefer to draw the
fermion lines all at once, as with the following example (note also that the way
vertices are named is completely different):

\begin{codeexample}[]
\feynmandiagram [horizontal=f2 to f3] {
  f1 -- [fermion] f2 -- [fermion] f3 -- [fermion] f4,
  f2 -- [photon] p1,
  f3 -- [photon] p2,
};
\end{codeexample}

As a final remark, the calculation of where vertices should be placed is usually
done through an algorithm written in Lua.  As a result, \LuaTeX{} is required in
order to make use of these algorithms.  If \LuaTeX{} is not used,
\tikzfeynmanname{} will default to a more rudimentary algorithm and will warn
the user instead.

\subsection{Adding Styles}
\label{subsec:adding_styles}

So far, the examples have only used the |photon| and |fermion| styles.  The
\tikzfeynmanname{} package comes with quite a few extra styles for edges and
vertices which are all documented over in \cref{sec:documentation}.  For
example, it is possible to add momentum arrows with |momentum=|\meta{text}, and
in the case of end vertices, the particle can be labelled with
|particle=|\meta{text}.  This is illustrated in the following example where the
generic \(s\)-channel diagram from \cref{subsec:a_first_diagram} is annotated
into a \(e^{+}e^{-} \to \mu^{+}\mu^{-}\) diagram:
\begin{codeexample}[]
\feynmandiagram [horizontal=a to b] {
  i1 [particle=\(e^{-}\)] -- [fermion] a -- [fermion] i2 [particle=\(e^{+}\)],
  a -- [photon, edge label=\(\gamma\), momentum'=\(k\)] b,
  f1 [particle=\(\mu^{+}\)] -- [fermion] b -- [fermion] f2 [particle=\(\mu^{-}\)],
};
\end{codeexample}
In addition to the style keys documented below, style keys from \tikzname{} can
be used as well:
\begin{codeexample}[]
\feynmandiagram [horizontal=a to b] {
  i1 [particle=\(e^{-}\)] -- [fermion, very thick] a -- [fermion, opacity=0.2] i2 [particle=\(e^{+}\)],
  a -- [red, photon, edge label=\(\gamma\), momentum'={[arrow style=red]\(k\)}] b,
  f1 [particle=\(\mu^{+}\)] -- [fermion, opacity=0.2] b -- [fermion, very thick] f2 [particle=\(\mu^{-}\)],
};
\end{codeexample}
For a list of all the various styles that \tikzname{} provides, have a look at
the \pgfmanual; it is extremely thorough and provides many usage examples.

\tikzfeynmanname{} also supports combining styles together which can be useful
in certain models such as supersymmetry:
\begin{codeexample}[]
\feynmandiagram [horizontal=a to b] {
  i1 [particle=\(\tilde W\)] -- [plain, boson] a -- [anti fermion] i2 [particle=\(q\)],
  a -- [charged scalar, edge label=\(\tilde q\)] b,
  f1 [particle=\(\tilde g\)] -- [plain, gluon] b -- [fermion] [particle=\(q\)],
};
\end{codeexample}

\subsection{When the Algorithm Isn't Enough}
\label{subsec:when_the_algorithm_isnt_enough}

By default, the |\feynmandiagram| and |\diagram| commands use the
|spring layout| algorithm to place all the edges\footnote{For more details on
  this layout and any other graph layouts available, see
  \cref{subsubsec:diagram_keys}}. The |spring layout| algorithm attempts to
`spread out' the diagram as much as possible which---for most simpler
diagrams---gives a satisfactory result; however in some cases, this does not
produce the best diagram and this section will look at alternatives.  There are
three main alternatives:
\begin{description}
\item[Add invisible edges] While still using the default algorithm, it is
  possible to force certain vertices to be closer together by adding extra edges
  and making them invisible through |draw=none|.  The algorithm will treat these
  extra edges in the same way, but they are simply not drawn at the end;

\item[Use a different algorithm] In some circumstances, other algorithms may be
  better suited.  Some of the other graph layout algorithms are listed in
  \cref{subsubsec:diagram_keys}, and an exhaustive list of all algorithms and
  their parameters is given in the \pgfmanual;

\item[Manual placement] As a last resort, very complicated or unusual diagrams
  will require each vertex to be manually placed.  \tikzfeynmanname{} offers a
  simplified syntax for manual placement so that it is not too difficult.
\end{description}

\subsubsection{Invisible Edges}
\label{subsubsec:invisible_edges}

The underlying algorithm treats all edges in exactly the same way when
calculating where to place all the vertices, and the actual drawing of the
diagram (after the placements have been calculated) is done separately.
Consequently, it is possible to add edges to the algorithm, but prevent them
from being drawn by adding |draw=none| to the edge style.

This is particularly useful if you want to ensure that the initial or final
states remain closer together than they would have otherwise as illustrated in
the following example (note that |opacity=0.2| is used instead of |draw=none| to
illustrate where exactly the edge is located).

\begin{codeexample}[]
% No invisible to keep the two photons together
\feynmandiagram [small, horizontal=a to t1] {
  a [particle=\(\pi^{0}\)] -- [scalar] t1 -- t2 -- t3 -- t1,
  t2 -- [photon] p1 [particle=\(\gamma\)],
  t3 -- [photon] p2 [particle=\(\gamma\)],
};
\end{codeexample}
\begin{codeexample}[]
% Invisible edge ensures photons are parallel
\feynmandiagram [small, horizontal=a to t1] {
  a [particle=\(\pi^{0}\)] -- [scalar] t1 -- t2 -- t3 -- t1,
  t2 -- [photon] p1 [particle=\(\gamma\)],
  t3 -- [photon] p2 [particle=\(\gamma\)],
  p1 -- [opacity=0.2] p2,
};
\end{codeexample}

\subsubsection{Alternative Algorithms}
\label{subsubsec:alternative_algorithms}

The graph drawing library from \tikzname{} has several different algorithms to
position the vertices\footnote{See \cref{subsubsec:diagram_keys} for some
  alternative algorithms.}  By default, |\diagram| and |\feynmandiagram| use the
|spring layout| algorithm to place the vertices.  The |spring layout| attempts
to spread everything out as much as possible which, in most cases, gives a nice
diagram; however, there are certain cases where this does not work.  A good
example where the |spring layout| does not work are decays where we have the
decaying particle on the left and all the daughter particles on the right.
\begin{codeexample}[]
% Using the default spring layout
\feynmandiagram [horizontal=a to b] {
  a [particle=\(\mu^{-}\)] -- [fermion] b -- [fermion] f1 [particle=\(\nu_{\mu}\)],
  b -- [boson, edge label=\(W^{-}\)] c,
  f2 [particle=\(\overline \nu_{e}\)] -- [fermion] c -- [fermion] f3 [particle=\(e^{-}\)],
};
\end{codeexample}
\begin{codeexample}[]
% Using the layered layout
\feynmandiagram [layered layout, horizontal=a to b] {
  a [particle=\(\mu^{-}\)] -- [fermion] b -- [fermion] f1 [particle=\(\nu_{\mu}\)],
  b -- [boson, edge label'=\(W^{-}\)] c,
  c -- [anti fermion] f2 [particle=\(\overline \nu_{e}\)],
  c -- [fermion] f3 [particle=\(e^{-}\)],
};
\end{codeexample}
You may notice that in addition to adding the |layered layout| style to
|\feynmandiagram|, we also changed the order in which we specify the vertices.
This is because the |layered layout| algorithm does pay attention to the order
in which vertices are declared (unlike the default |spring layout|); as a
result, |c--f2, c--f3| has a different meaning to |f2--c--f3|.  In the former
case, |f2| and |f3| are both on the layer below |c| as desired; whilst the
latter case places |f2| on the layer above |c| (that, the same layer as where
the \(W^{-}\) originates).

\subsubsection{Manual Placement}
\label{subsubsec:manual_placement}

In more complicated diagrams, it is quite likely that none of the algorithms
work, no matter how many invisible edges are added.  In such cases, the vertices
have to be placed manually.  \tikzfeynmanname{} allows for vertices to be
manually placed by using the |\vertex| command.

The |\vertex| command is available only within the |feynman| environment (which
itself is only available inside a |tikzpicture|).  The |feynman| environment
loads all the relevant styles from \tikzfeynmanname{} and declares additional
\tikzfeynmanname-specific commands such as |\vertex| and |\diagram|.  This is
inspired from PGFPlots and its use of the |axis| environment.

The |\vertex| command is very much analogous to the |\node| command from
\tikzname{}, with the notable exception that the vertex contents are optional;
that is, you need not have |{|\meta{text}|}| at the end.  In the case where |{}|
is specified, the vertex automatically is given the |particle| style, and
otherwise it is a usual (zero-sized) vertex.

To specify where the vertices go, it is possible to give explicit coordinates
though it is easier to use the |positioning| library from \tikzname{} which
allows vertices to be placed relative to existing vertices\footnote{The
  \pgfmanual{} has extensive documentation explaining how to use the
  |positioning| library.}.  By using relative placements, it is possible to
easily tweak one part of the graph and everything will adjust accordingly---the
alternative being to manually adjust the coordinates of every affected vertex.

Finally, once all the vertices have been specified, the |\diagram*| command is
used to specify all the edges.  This works in much the same way as |\diagram|
(and also |\feynmandiagram|), except that it uses an \emph{very basic} algorithm
to place new nodes and allows existing (named) nodes to be included.  In order
to refer to an existing node, the node must be given in parentheses.

This whole process of specifying the nodes and then drawing the edges between
them is shown below for the muon decay:

\begin{codeexample}[]
\begin{tikzpicture}
  \begin{feynman}
    \vertex (a) {\(\mu^{-}\)};
    \vertex [right=of a] (b);
    \vertex [above right=of b] (f1) {\(\nu_{\mu}\)};
    \vertex [below right=of b] (c);
    \vertex [above right=of c] (f2) {\(\overline \nu_{e}\)};
    \vertex [below right=of c] (f3) {\(e^{-}\)};

    \diagram* {
      (a) -- [fermion] (b) -- [fermion] (f1),
      (b) -- [boson, edge label'=\(W^{-}\)] (c),
      (c) -- [anti fermion] (f2),
      (c) -- [fermion] (f3),
    };
  \end{feynman}
\end{tikzpicture}
\end{codeexample}

One important note is that it is not possible to arbitrarily use automatic and
manual placement of vertices; automatic placement should be used first (if at
all) and manually placed vertices should be specified subsequently.  This is due
to most algorithms not being able to handle nodes with fixed placement.  The
following example illustrates how an initial automatically generated layout can
be built upon by placing new nodes relative to the automatically placed ones:

\begin{codeexample}[]
\begin{tikzpicture}
  \begin{feynman}
    % The first decay layout is done automatically
    \diagram [horizontal'=i to a] {
      i -- [scalar] a,
      b1 -- [fermion] a -- [fermion] b2,
      b1 -- [draw=none] b2,
    };
    % New vertices are added relative to automatically placed ones
    \vertex [above right=of b1] (c1);
    \vertex [right=of b1] (c2);
    \vertex [right=of b2] (c3);
    \vertex [below right=of b2] (c4);
    % and they are linked together here
    \diagram* {
      (b1) -- [gluon] (c1),
      (b1) -- [anti fermion] (c2),
      (b2) -- [fermion] (c3),
      (b2) -- [gluon] (c4),
    };
  \end{feynman}
\end{tikzpicture}
\end{codeexample}

\clearpage
\section{Documentation}
\label{sec:documentation}

\subsection{Commands \emph{\&} Environments}
\label{subsec:commands}

\begin{command}{\tikzfeynmanset\marg{options}}
  This command will process \meta{options} using |\pgfkeys| with the default
  path set to |/tikzfeynman|.  Typically, \meta{options} will be a
  comma-separated list of the form \meta{key}|=|\meta{value}, though the full
  power of the mechanism behind |\pgfkeys| can be used (see the \pgfmanual{} for
  a complete description).

  Typically, this is used in the preamble of the document to add or change
  certain keys for the whole document.
\end{command}

\begin{command}{\feynmandiagram\opt{\oarg{TikZ options}\oarg{diagram options}}\marg{diagram instructions}}
  This command creates a |{tikzpicture}| and |{feynman}| environment, and
  places a |\diagram| inside with the provided \meta{diagram instruction}.
  Please refer to the documentation for |\diagram| for the \meta{diagram
    instruction} syntax.

  The optional arguments specified in \meta{TikZ options} are passed on to the
  |{tikzpicture}|, and the \meta{diagram options} are passed on to |\diagram|.
  If only one optional argument is given, then the optional arguments are given
  to both.  A single optional argument will usually suffice as most keys are
  recognized by both commands and appropriately selected; however, in the event
  that a key is not recognized, both options are provided.
\end{command}

\begin{environment}{{feynman}\opt{\oarg{options}}}
  The |{feynman}| environment is where all the drawing of Feynman diagrams takes
  place.  It makes all the \tikzfeynmanname{} styles available and defines
  commands such as |\vertex| and |\diagram| which are otherwise unavailable
  outside of this environment.  The |{feynman}| environment is only accessible
  within the |{tikzpicture}| environment.

  Options which are passed in \meta{options} apply for the whole environment in
  the same way that the |{scope}| environment work in \tikzname.

  \begin{command}{\vertex\opt{\oarg{options}} (\meta{name}) \opt{at (\meta{coordinate})} \opt{\marg{contents}};}
    Defines a new vertex with the provided \meta{name}.  If \meta{contents} is
    not provided, the resulting vertex will have zero size.  On the other hand,
    if \meta{contents} is provided, the |particle=|\meta{contents} style is
    applied.  Additional styles can be applied to the vertex through
    \meta{options}.

    The final semicolon (|;|) is vital for this command since without it, the
    \LaTeX{} engine will not know when the |\vertex| command ends.
    Additionally, this command \emph{cannot} be chained like one can do with the
    inbuilt \tikzname{} commands.

    This command is only available with the |{feynman}| environment.
  \end{command}

  \begin{command}{\diagram\opt{\oarg{options}}\marg{diagram instructions}}
    Begins a new diagram using the |spring layout|.  Keys passed through
    \meta{options} can include general \tikzname{} keys, graph-specific keys and
    any applicable \tikzfeynmanname{} keys too.  Other algorithms (such as |tree
    layout|) can be passed through \meta{options} and that will override the
    |spring layout|.

    The syntax for the \meta{diagram instructions} is thoroughly described in
    the \pgfmanual, but in the context of this package, it will usually suffice
    to know the following:
    \begin{itemize}
    \item Vertices within the graph are specified with no delimiters (i.e.~no
      parenthesis, no brackets) and only require spaces around either side.  In
      order to refer to a vertex defined outside of the |\diagram| command, its
      name must be given in parenthesis: |(|\meta{name}|)|.  Note that in order
      to refer to external vertices, one must use |\diagram*| as most algorithms
      (including the default |spring layout|) are incompatible with vertices
      defined outside of the algorithm.

      When a vertex name is used multiple times, the underlying algorithm will
      consider them to be the same vertex and introduces additional edges.

      Options can be given to the vertex in brackets after the name: \meta{name}
      |[|\meta{options}|]|.  For vertices defined outside of the |\diagram|
      command, these options should be specified when the vertex is first
      declared.

    \item The edges between each pair of vertices is specified with |--|, and
      these can be chained together: \verb|a -- b -- c|.  In order to pass a
      style to the edge, it is specified in brackets after the dashed:
      |-- | \oarg{options}.  For example, to make on edge red, one would use
      |-- [red]|.

    \item A comma (|,|)---or equivalently a semicolon (|;|)---specifies the end
      of a sequence of edges and vertices and allows for another sequence to be
      started.  So \verb|a -- b, c -- d| will create two disconnected edges.

    \item Subgroups (akin to scopes in \tikzname) are specified with braces:
      |{|\oarg{options}\meta{diagram instructions}|}|.  This can be quite
      useful when a lot of edges or nodes share a common style.  For example,
      one could use |{[edges={fermion}]|\verb| a -- b -- c, x -- y -- z}| and
      every edge will have the |fermion| style applied automatically.

      Another useful feature of subgroups is that an edge to a group will create
      an edge to each vertex in that subgroup as shown below.  The example also
      shows how they can be nested which in some cases (such as with a
      |layered layout|) can be extremely useful.

\begin{codeexample}[]
\feynmandiagram [nodes=circle, horizontal=a1 to b3] {
  a1 -- {b1, b2, b3 -- {c1, c2 -- d1}}
};
\end{codeexample}
    \end{itemize}
  \end{command}

  \begin{command}{\diagram*\opt{\oarg{options}}\marg{diagram instructions}}
    Same as |\diagram|, but instead of using the |spring layout| algorithm to
    place the vertices, this uses the most basic algorithm.  This basic
    algorithm in most cases will not produce a satisfactory diagram, but is
    intended to be used with vertices which have been declared and positioned
    outside of the |\diagram*| command.  Essentially, |\diagram*| should be used
    only to connect existing vertices.
  \end{command}
\end{environment}



\subsection{Keys \emph{\&} Styles}
\label{subsec:Keys}

The various styles and options that allow the Feynman diagrams to be customized
are defined in what \tikzname{} calls \emph{keys}.  The documentation includes
all keys which are defined within \tikzfeynmanname{} which all begin with the
prefix |/tikzfeynman|.  In addition, a few of the keys from \tikzname{} itself
which are particularly useful to \tikzfeynmanname{} are documented and these are
prefixed with |/tikz| or |/graph drawing|.  Please refer to the \pgfmanual{} for
a more in thorough documentation of the \tikzname{} keys.


\tikzfeynmanname{} provides many |every |\meta{key} which allow every instance
of \meta{key} to be modified.  For example, to make every diagram red except for
small diagrams which should be blue, then one would add to the preamble:
\begin{codeexample}[execute code=false]
\tikzfeynmanset{
  every diagram={red},
  every small={blue},
}
\end{codeexample}
If you are completely unhappy with a particular inbuilt style, you can define
your own style with \meta{key}|/.style={...}| as shown in the following example:
\begin{codeexample}[execute code=false]
\tikzfeynmanset{
  myblob/.style={
    shape=circle,
    draw=blue,
    fill=red,
  }
}
\end{codeexample}

All the keys defined here are made available inside the |{feynman}| environment
and inside |\feynmandiagram|; but if you wish to access them outside of this
(say, in a regular |{tikzpicture}| environment), you will need to specify the
full path with the leading |/tikzfeynman|.

\subsubsection{Feynman Keys}
\label{subsubsec:feynman_keys}

\begin{keylist}{%
    /tikzfeynman/execute at begin feynman=\marg{\TeX{} code},
    /tikzfeynman/execute at end feynman=\marg{\TeX{} code}}
  Allows for custom code to be executed at the start or end of each |{feynman}|
  environment.
\end{keylist}

\begin{key}{/tikzfeynman/every feynman=\meta{styles}}
  Set of styles which are applied to every |{feynman}| environments (and
  consequently, every apply inside all |\feynmandiagram| too).  The style also
  applies to regular \tikzname{} commands used inside the |{feynman}|
  environment.

\begin{codeexample}[]
\tikzfeynmanset{every feynman={red}}
\begin{tikzpicture}
  \node at (0, 0.5) {This is not red};
  \begin{feynman}
    \node at (0, -0.5) {This is red};
  \end{feynman}
\end{tikzpicture}
\end{codeexample}
\end{key}

\begin{stylekey}{/tikzfeynman/inline=\meta{node}}
  A style used to display a Feynman diagram inline (typically in an equation),
  and aligning such that its vertical placement is at the node specified.  The
  node specification must be enclosed in parentheses.  For nodes which contain
  text (such as when the |particle| style is applied), it is possible to use the
  baseline of the text inside the node to line up with the baseline of the
  equation by using |(|\meta{node}|.base)| as demonstrated in the following
  example.  Note that this key applies additional styles to make the diagram fit
  in an equation more nicely; if you do not wish to have these additional
  styles, use the |baseline| key.

  \begin{equation}
    \feynmandiagram [inline=(d.base), horizontal=d to b] {
      a -- [fermion] b -- [fermion] c,
      b -- [boson] d [particle=\(\gamma\)],
    };
    = i g_{e} \gamma^{\mu}
  \end{equation}

\begin{codeexample}[execute code=false]
\begin{equation}
  \feynmandiagram [inline=(d.base), horizontal=d to b] {
    a -- [fermion] b -- [fermion] c,
    b -- [boson] d [particle=\(\gamma\)],
  };
  = i g_{e} \gamma^{\mu}
\end{equation}
\end{codeexample}
\end{stylekey}

\begin{key}{/tikz/baseline=\meta{node}}
  Changes the vertical alignment of the Feynman diagram such that it diagram's
  baseline is at the node specified.  This works in the same way as
  |inline=|\meta{node}, but it does not apply any additional styles (notice how
  the following example is larger than the one above).
  \begin{equation}
    \feynmandiagram [baseline=(d.base), horizontal=d to b] {
      a -- [fermion] b -- [fermion] c,
      b -- [boson] d [particle=\(\gamma\)],
    };
    = i g_{e} \gamma^{\mu}
  \end{equation}

\begin{codeexample}[execute code=false]
\begin{equation}
  \feynmandiagram [baseline=(d.base), horizontal=d to b] {
    a -- [fermion] b -- [fermion] c,
    b -- [boson] d [particle=\(\gamma\)],
  };
  = i g_{e} \gamma^{\mu}
\end{equation}
\end{codeexample}
\end{key}

\begin{keylist}{%
    /graph drawing/horizontal=\meta{node} to \meta{node},
    /graph drawing/horizontal'=\meta{node} to \meta{node},
    /graph drawing/vertical=\meta{node} to \meta{node},
    /graph drawing/vertical'=\meta{node} to \meta{node}}
  The underlying algorithm will arrange all the nodes relative to each other,
  but beyond that it has no idea how the overall graph should be oriented.
  By using one of the above keys, the final output of the algorithm is oriented
  and/or mirrored such that the two nodes specified are on the same horizontal
  (or vertical) line.  The two nodes need not actually be connected by an edge
  for this to work.

  The two \meta{node} specifications should \emph{not} be enclosed in
  parentheses, unlike the |inline| and |baseline| keys.

  The |horizontal'| and |vertical'| keys work in the same was as |horizontal|
  and |vertical|, but with a flip.

\begin{codeexample}[]
\feynmandiagram [inline=(b), horizontal=a to b, red] {
  a -- b -- {c [particle=\(c\)], d [particle=\(d\)]}
};
\feynmandiagram [inline=(b), horizontal'=a to b, blue] {
  a -- b -- {c [particle=\(c\)], d [particle=\(d\)]}
};
\feynmandiagram [inline=(b), vertical=a to b, green!40!black] {
  a -- b -- {c [particle=\(c\)], d [particle=\(d\)]}
};
\feynmandiagram [inline=(b), vertical=b to a, black] {
  a -- b -- {c [particle=\(c\)], d [particle=\(d\)]}
};
\end{codeexample}
\end{keylist}

\subsubsection{Diagram Keys}
\label{subsubsec:diagram_keys}

\begin{key}{/tikzfeynman/every diagram=\meta{styles}}
  Set of styles which are applied to every diagram; that is, to everything
  inside the |\feynmandiagram|, |\diagram| and |\diagram*| commands but not the
  general |{feynman}| environment (see \cref{subsubsec:feynman_keys} for that).

\begin{codeexample}[]
\tikzfeynmanset{every diagram={red}}
\feynmandiagram [small, horizontal=d to b] {
  a -- [fermion] b -- [fermion] c,
  b -- [boson] d,
};
\end{codeexample}
\end{key}

\begin{keylist}{%
    /tikzfeynman/small,
    /tikzfeynman/medium,
    /tikzfeynman/large}
  Changes the default separation between the vertices and changes the size of
  arrows, blobs, and other shapes to fit different context.  The |small| size is
  best used with when the diagram is quite simple and does not have too many
  annotations (such as momentum arrows and particle labels).  The |medium| size
  is the default and is usually large enough that even diagrams with many labels
  and momentum arrows do not become too cluttered.  Finally the |large| key is
  best for large illustrations as used on the title page of this document.

\begin{codeexample}[]
\feynmandiagram [baseline=(b), small, horizontal=d to b, red] {
  a -- [fermion] b -- [fermion] c,
  b -- [boson] d,
};
\feynmandiagram [baseline=(b), medium, horizontal=d to b, green!40!black] {
  a -- [fermion] b -- [fermion] c,
  b -- [boson] d,
};
\feynmandiagram [baseline=(b), large, horizontal=d to b, blue] {
  a -- [fermion] b -- [fermion] c,
  b -- [boson] d,
};
\end{codeexample}
\end{keylist}

There are several algorithms which are available to place the vertices which are
all provided within the |graph drawing| library from \tikzname.  Below are
listed a few of these layouts which are more relevant for drawing Feynman
diagrams.  For a more complete description of how these algorithms work, please
refer to the \pgfmanual{}.  Please note that most graph drawing algorithms are
implemented in Lua and thus require \LuaTeX{} in order to work.  If \LuaTeX{} is
not used, \tikzfeynmanname{} will default to a more rudimentary algorithm and
will warn the user instead.

\begin{key}{/graph drawing/spring layout=\opt{\meta{string}}}
  Uses Hu's spring layout \cite{hu2005} as implemented by Pohlmann
  \cite{pohlmann2011}.  This models each edge as springs and attempts to spread
  everything out as much as possible.  This is the default layout.

\begin{codeexample}[]
\feynmandiagram [nodes=circle, small, horizontal=c to d] {
  {a, b} -- c -- d -- {e, f},
};
\end{codeexample}
\end{key}

\begin{key}{/graph drawing/spring electrical layout=\opt{\meta{string}}}
  Uses Hu's spring electrical layout \cite{hu2005} as implemented by Pohlmann
  \cite{pohlmann2011}.  This models each edge as springs and gives each vertex a
  charge.  This algorithm allows for the charge of a particular vertex to be
  adjusted using the |electric charge| key (the default is |1|).

\begin{codeexample}[]
\feynmandiagram [nodes=circle,
    small, horizontal=c to d,
    spring electrical layout
  ] {
  {a, b [electric charge=2]} -- c -- d -- {e, f [electric charge=0.1]},
};
\end{codeexample}
\end{key}

\begin{key}{/graph drawing/layered layout=\opt{\meta{string}}}
  Uses the Sugiyama layout algorithm \cite{eades1991} as implemented by Pohlmann
  \cite{pohlmann2011} in order to place the node.

  When an edge is specified, the first vertex is always located on the layer
  above the second vertex.  This creates a hierarchy of vertices which is
  particularly useful for decays.

  Two vertices can be forced to be on the same layer with the
  |/graph drawing/same layer| key.

\begin{codeexample}[]
\feynmandiagram [nodes=circle, small, horizontal=a to b, layered layout] {
  a -- b -- {c, d -- {e, f}},
};
\end{codeexample}
\begin{codeexample}[]
\feynmandiagram [nodes=circle, small, horizontal=a to b, layered layout] {
  a -- b -- {c -- {c1, c2}, d -- {d1, d2}},
  {[same layer] c1, d},
};
\end{codeexample}
\end{key}

\begin{key}{/graph drawing/tree layout=\opt{\meta{string}}}
  Uses the Reingold--Tilford algorithm in order to place the node.  This works
  in a similar way to the layered layout, but has quite a lot of additional
  options to handle missing children in the tree.  Please refer to the
  \pgfmanual{} for a thorough description of these additional features.

  When an edge is specified, the first vertex is always located on the layer
  above the second vertex.  This creates a hierarchy of vertices which is
  particularly useful for decays.

\begin{codeexample}[]
\feynmandiagram [nodes=circle, small, horizontal=a to b, tree layout] {
  a -- b -- {c, d -- {e, f}},
};
\end{codeexample}
\begin{codeexample}[]
\feynmandiagram [nodes=circle, small, horizontal=a to b, tree layout] {
  a -- b -- {c -- {c1, c2}, d -- {d1, d2}},
};
\end{codeexample}
\end{key}

\begin{keylist}{%
  /tikz/graphs/edges=\meta{options},
  /tikz/graphs/nodes=\meta{options}}
  Just like it is possible to change the shape of every vertex or edge in the
  whole document, it is also possible to change the shape of every vertex or
  edge in a single diagram by modifying these keys and adding the desired
  styles.
\end{keylist}

\subsubsection{Vertex Keys}
\label{subsubsec:vertex_keys}

\begin{key}{/tikzfeynman/vertex}
  The default, base style applied to every vertex initially.  Other styles
  are subsequently added.  This sets the vertex |shape| to be a |coordinate|,
  that is, a null shape with no size nor width.
\end{key}

\begin{key}{/tikzfeynman/every \meta{vertex shape}=\meta{styles}}
  The style of specific vertices can be modified by changing the appropriate
  |every |\meta{vertex shape} key.  For example, in order to change the style of
  every |dot|-styled vertex, you can modify the |every dot| key, or to modify
  every vertex globally, the |every vertex| key can be modified.

\begin{codeexample}[]
\tikzfeynmanset{
  every vertex={red, dot},
  every particle={blue},
  every blob={draw=green!40!black, pattern color=green!40!black},
}
\feynmandiagram [horizontal=a to b] {
  a [particle={\(\gamma, Z\)}] -- [boson] b [blob],
  c -- [fermion] b -- [fermion] d,
};
\end{codeexample}
\end{key}

\begin{key}{/tikzfeynman/dot}
  Modifies the vertex so that it has a small filled circle.

\begin{codeexample}[]
\feynmandiagram [small] {
  a -- b [dot] -- {c, d}
};
\end{codeexample}
\end{key}

\begin{key}{/tikzfeynman/square dot}
  Modifies the vertex so that it has a small filled square.

\begin{codeexample}[]
\feynmandiagram [small] {
  a -- b [square dot] -- {c, d}
};
\end{codeexample}
\end{key}

\begin{key}{/tikzfeynman/empty dot}
  Modifies the vertex so that it has a small empty circle.

\begin{codeexample}[]
\feynmandiagram [small] {
  a -- b [empty dot] -- {c, d}
};
\end{codeexample}
\end{key}

\begin{key}{/tikzfeynman/crossed dot}
  Modifies the vertex so that it has a small circle with a cross inside.

\begin{codeexample}[]
\feynmandiagram [small] {
  a -- b [crossed dot] -- {c, d}
};
\end{codeexample}
\end{key}

\begin{key}{/tikzfeynman/blob}
  Modifies the vertex so that it is a large blob, usually used to denote an
  effective operator.

\begin{codeexample}[]
\feynmandiagram [small] {
  a -- b [blob] -- {c, d}
};
\end{codeexample}
\end{key}

\begin{key}{/tikzfeynman/particle=\meta{name}}
  Modifies the vertex so that it shows \meta{name}.  This is intended to label
  initial and final particles, but it should not be used on internal vertices as
  it will result in the lines at the vertex having a gap.  For propagators the
  |edge label| key is much more appropriate.

  Note that if \meta{name} contains characters such as brackets (|[]|) or
  commas (|,|), the whole \meta{name} has to be enclosed in braces (|{}|);
  otherwise, the parser will interpret the comma as the end of the \meta{name}
  and the start of the next key, or the closing bracket as the end of all
  optional arguments.

\begin{codeexample}[]
\feynmandiagram [small, horizontal=a to b] {
  a [particle={\(\gamma, Z\)}] -- [boson] b -- {c, d},
};
\end{codeexample}
\end{key}


\subsubsection{Edge Keys}
\label{subsubsec:edge_keys}

Just like with the various vertex keys, each edge type has a corresponding
|every |\meta{edge type}; however, due to the existence of very similar keys
such as |scalar|, |charged scalar| and |anti charged scalar|, more specific keys
inherit styles from less specific ones.  For example, styles in |every scalar|
will apply to |charged scalar| and |anti charged scalar|.

The edge styles do not override each other and instead stack.  This means that
an edge styled with both |boson| and |fermion| will consist of a straight line
with a wavy line on top and an error.  If no styles are provided, then a simple
solid line is drawn.

\begin{key}{/tikzfeynman/every edge=\meta{styles}}
  A style to apply to every edge initially.
\end{key}

\begin{stylekey}{/tikzfeynman/every \meta{edge style}=\meta{styles}}
  The style of specific edges can be modified by changing the appropriate
  |every |\meta{edge style} key.  For example, in order to make a global change
  to every |boson|, you can modify the |every boson| key.

\begin{codeexample}[]
\tikzfeynmanset{
  every edge={green},
  every boson={red},
  every photon={blue},
}
\feynmandiagram [small] {
  a [particle=\(a\)] -- [boson] o -- [photon] b [particle=\(b\)],
  f1 [particle=\(c\)] -- [fermion] o -- [scalar] f2 [particle=\(d\)],
};
\end{codeexample}
\end{stylekey}

\begin{key}{/tikzfeynman/plain}
  Draws a simple solid line.  This style is identical to the default style
  applied if no other style is used, but is provided so that it can be combined
  with other styles.

\begin{codeexample}[]
\feynmandiagram [horizontal=a to b] {a -- [plain] b};
\feynmandiagram [horizontal=a to b] {a -- [plain, gluon] b};
\end{codeexample}
\end{key}

\begin{key}{/tikzfeynman/boson}
  Draws a sinusoidal line to denote a boson.

\begin{codeexample}[]
\feynmandiagram [horizontal=a to b] {a -- [boson] b};
\end{codeexample}
\end{key}

\begin{key}{/tikzfeynman/charged boson}
  Draws a sinusoidal line with an arrow to denote a charged boson.

\begin{codeexample}[]
\feynmandiagram [horizontal=a to b] {a -- [charged boson] b};
\end{codeexample}
\end{key}

\begin{key}{/tikzfeynman/anti charged boson}
  Draws a sinusoidal line with an arrow pointing the other way to denote a
  anti charged boson.

\begin{codeexample}[]
\feynmandiagram [horizontal=a to b] {a -- [anti charged boson] b};
\end{codeexample}
\end{key}

\begin{key}{/tikzfeynman/photon}
  Draws a sinusoidal line to denote a photon.

\begin{codeexample}[]
\feynmandiagram [horizontal=a to b] {a -- [photon] b};
\end{codeexample}
\end{key}

\begin{key}{/tikzfeynman/scalar}
  Draws a dashed line to denote a scalar.

\begin{codeexample}[]
\feynmandiagram [horizontal=a to b] {a -- [scalar] b};
\end{codeexample}
\end{key}

\begin{key}{/tikzfeynman/charged scalar}
  Draws a dashed line with an arrow to denote a charged scalar.

\begin{codeexample}[]
\feynmandiagram [horizontal=a to b] {a -- [charged scalar] b};
\end{codeexample}
\end{key}

\begin{key}{/tikzfeynman/anti charged scalar}
  Draws a dashed line with an arrow pointing the other way to denote a charged
  scalar antiparticle.

\begin{codeexample}[]
\feynmandiagram [horizontal=a to b] {a -- [anti charged scalar] b};
\end{codeexample}
\end{key}

\begin{key}{/tikzfeynman/ghost}
  Draws a dotted line to denote a ghost.

\begin{codeexample}[]
\feynmandiagram [horizontal=a to b] {a -- [ghost] b};
\end{codeexample}
\end{key}


\begin{key}{/tikzfeynman/fermion}
  Draws a solid line with an arrow to denote a fermion.

\begin{codeexample}[]
\feynmandiagram [horizontal=a to b] {a -- [fermion] b};
\end{codeexample}
\end{key}

\begin{key}{/tikzfeynman/anti fermion}
  Draws a solid line with an arrow pointing the other way to denote an antifermion.

\begin{codeexample}[]
\feynmandiagram [horizontal=a to b] {a -- [anti fermion] b};
\end{codeexample}
\end{key}

\begin{key}{/tikzfeynman/majorana}
  Draws a solid line with two arrows pointing to the center to denote a Majorana
  particle.

\begin{codeexample}[]
\feynmandiagram [horizontal=a to b] {a -- [majorana] b};
\end{codeexample}
\end{key}

\begin{key}{/tikzfeynman/anti majorana}
  Draws a solid line with two arrows pointing to the ends to denote a Majorana
  particle.

\begin{codeexample}[]
\feynmandiagram [horizontal=a to b] {a -- [anti majorana] b};
\end{codeexample}
\end{key}

\begin{key}{/tikzfeynman/gluon}
  Draws a coiled line to denote a gluon.

\begin{codeexample}[]
\feynmandiagram [horizontal=a to b] {a -- [gluon] b};
\end{codeexample}
\end{key}

\begin{keylist}{%
  /tikz/edge label=\meta{text},
  /tikz/edge label'=\meta{text}}
  Places a label halfway along the edge with the given text.  The primed key
  switches which side of the edge the label is placed.
\end{keylist}

\begin{key}{/tikzfeynman/insertion=\opt{\oarg{options}}\meta{distance}}
  Places an insertion (for mass or momentum insertion) along an edge.  The
  distance specifies how far along the edge the insertion should be placed such
  that |0| and |1| respectively correspond to the start and the end of the edge.

  Multiple insertions can be placed along a single edge by repeating the style
  key.

  Through the \meta{options} argument, the insertion size and style can be
  changed.

\begin{codeexample}[]
\feynmandiagram [horizontal=a to b] {a -- [insertion=0.33, insertion=0.67] b};
\end{codeexample}

  \begin{key}{/tikzfeynman/insertion/size=\meta{distance} (initially 3pt)}
    Specifies how big the insertion should be.  The length of each edge starting
    from the center will be \(\sqrt{2} \times \meta{distance}\).

\begin{codeexample}[]
\feynmandiagram [horizontal=a to b] {a -- [insertion={[size=10pt]0.4}] b};
\end{codeexample}
  \end{key}

  \begin{key}{/tikzfeynman/insertion/style=\meta{distance} (initially \normalfont empty)}
    Specifies additional styles to applying to the lines of the insertion.

\begin{codeexample}[]
\feynmandiagram [horizontal=a to b] {a -- [insertion={[style=red]0.4}] b};
\end{codeexample}
  \end{key}
\end{key}

\subsubsection{Momentum Keys}
\label{subsubsec:momentum_keys}

\begin{keylist}{%
    /tikzfeynman/momentum=\opt{\oarg{options}}\meta{text},
    /tikzfeynman/momentum'=\opt{\oarg{options}}\meta{text},
    /tikzfeynman/reversed momentum=\opt{\oarg{options}}\meta{text},
    /tikzfeynman/reversed momentum'=\opt{\oarg{options}}\meta{text},
    /tikzfeynman/rmomentum=\opt{\oarg{options}}\meta{text},
    /tikzfeynman/rmomentum'=\opt{\oarg{options}}\meta{text}}
  Places a momentum arrow on the specified edge with label given by \meta{text}.
  The primed (|'|) version places the momentum arrow on the other side of the
  edge; that is, if the momentum arrow was on the right, it will be placed on
  the left of the edge.  The |reversed momentum| and |reversed momentum'| keys
  are analogous to |momentum| and |momentum'| except that the momentum arrow
  points in the opposite direction.  Finally, the |rmomentum| and |rmomentum'|
  are aliases of |reversed momentum| and |reversed momentum'|.

  Note that due to the way the arrow is drawn, it does not inherit styles of the
  edge.  As a result, they have to be re-specified through \meta{options}.

\begin{codeexample}[]
\feynmandiagram [layered layout, horizontal=a to b] {
  a -- [red, fermion, edge label'=\(ab\), momentum={[arrow style=red]\(p_{ab}\)}] b
    -- [blue, photon, edge label'=\(bc\)] c
    -- [green!40!black, scalar, momentum=\(p_{cd}\)] d,
};
\end{codeexample}

\begin{codeexample}[]
\feynmandiagram [layered layout, horizontal=b to c] {
  a -- [photon, momentum=\(p\)] b
    -- [fermion, half left, looseness=1.5, momentum=\(k\)] c
    -- [fermion, half left, looseness=1.5, momentum=\(k-p\)] b,
  c -- [photon, momentum=\(p\)] d,
};
\end{codeexample}

The \meta{options} allows for the following options to customize the appearance
of the momentum arrows.  These can be either set globally with the usual
|\tikzfeynmanset|, or can be set on a individual basis through the
\meta{options} argument of the |momentum| key.  In the latter case, only the
\meta{key} in |/tikzfeynman/momentum/|\meta{key} is required.

\begin{key}{/tikzfeynman/momentum/label distance=\meta{distance} (initially 0pt)}
  Set the separation between the text and the arrow.  Note that the text is
  still surrounded by an |inner sep=0.3333em| by default so the default
  distance of |0pt| will not result in the momentum label touching the arrow.
\end{key}

\begin{key}{/tikzfeynman/momentum/arrow distance=\meta{distance} (initially 3mm)}
  Set the separation between the edge and the arrow.
\end{key}

\begin{key}{/tikzfeynman/momentum/arrow shorten=\meta{distance} (initially 0.15)}
  Specify the fraction of the total edge length by which the momentum arrow is
  shortened at each end.
\end{key}

\begin{key}{/tikzfeynman/momentum/label style=\meta{style} (initially \normalfont empty)}
  Define styles to apply to the momentum label node.
\end{key}

\begin{key}{/tikzfeynman/momentum/arrow style=\meta{style} (initially \normalfont empty)}
  Define style to apply to the momentum arrow.
\end{key}
\end{keylist}

\subsubsection{Modifier Keys}
\label{subsubsec:modifier_keys}

Modifier keys serve only to slightly modify a small feature of the edge.

\begin{keylist}{%
    /tikzfeynman/half left,
    /tikzfeynman/half right,
    /tikzfeynman/quarter left,
    /tikzfeynman/quarter right}
  Modifies the edge so that it bends left or right in such a way that it
  completes half a circle, or a quarter of a circle.

\begin{codeexample}[]
\feynmandiagram [horizontal=a to b] {
  a -- [red, fermion, half left] b -- [blue, fermion, half left] a,
};
\end{codeexample}

\begin{codeexample}[]
\feynmandiagram [horizontal=a to c] {
  a -- [red!0!blue, fermion, quarter left] b
    -- [red!33!blue, fermion, quarter left] c
    -- [red!66!blue, fermion, quarter left] d
    -- [red!100!blue, fermion, quarter left] a,
};
\end{codeexample}
\end{keylist}

\begin{keylist}{%
  /tikz/out=\meta{angle},
  /tikz/in=\meta{angle}}
  Specifies the angle at which the edge leaves the first vertex in an edge and
  the angle at which it enters the second vertex in an edge.
\end{keylist}

\begin{key}{/tikz/relative=\opt{\meta{true or false}} (default true)}
  If |relative| is set to |false|, the angle is relative to the paper whilst when
  |relative| is set to |true|, the angle is relative to the straight line
  joining the two vertices.
\end{key}

\begin{key}{/tikz/looseness=\meta{number} (initially 1)}
  As the name suggests, this specifies how `loose' or `tight' a curve is
  connecting two vertices.
\end{key}

\clearpage
\section{Examples}
\label{sec:examples}

Below are a few diagrams which demonstrate how the package can be used in some
more practical examples..

\begin{description}
\item[Vertex Rule] \hspace*{0pt} \newline
\begin{codeexample}[]
\feynmandiagram [horizontal=a to b] {
  a [particle=\(Z\)] -- [photon, momentum=\(p_{1}\)] b,
  f1 [particle=\(\overline f\)]
     -- [fermion, rmomentum'=\(p_{3}\)] b
     -- [fermion, momentum=\(p_{2}\)] f2 [particle=\(f\)],
};
\end{codeexample}

\item[Tree Level Diagrams] \hspace*{0pt} \newline
\begin{codeexample}[]
\feynmandiagram [horizontal=a to b] {
  i1 [particle=\(e^{-}\)] -- [fermion] a -- [fermion] i2 [particle=\(e^{+}\)],
  a -- [photon, edge label=\(\gamma\)] b,
  f1 [particle=\(\mu^{-}\)] -- [fermion] b -- [fermion] f2 [particle=\(\mu^{+}\)],
};
\end{codeexample}

\begin{codeexample}[]
\feynmandiagram [vertical'=a to b] {
  i1 [particle=\(e^{-}\)]
     -- [fermion] a
     -- [fermion] f1 [particle=\(e^{-}\)],
  a -- [photon, edge label=\(\gamma\)] b,
  i2 [particle=\(e^{+}\)]
     -- [anti fermion] b
     -- [anti fermion] f2 [particle=\(e^{+}\)],
};
\end{codeexample}

\begin{codeexample}[]
\begin{tikzpicture}
  \begin{feynman}
    \diagram [vertical'=a to b] {
      i1 [particle=\(e^{-}\)]
         -- [fermion] a
         -- [draw=none] f1 [particle=\(e^{+}\)],
      a -- [photon, edge label'=\(p\)] b,
      i2 [particle=\(e^{+}\)]
         -- [anti fermion] b
         -- [draw=none] f2 [particle=\(e^{-}\)],
    };
    \diagram* {
      (a) -- [fermion] (f2),
      (b) -- [anti fermion] (f1),
    };
  \end{feynman}
\end{tikzpicture}
\end{codeexample}

\item[Loops] \hspace*{0pt} \newline
\begin{codeexample}[]
\feynmandiagram [layered layout, horizontal=b to c] {
  a -- [photon, momentum=\(p\)] b
    -- [fermion, half left, momentum=\(k\)] c
    -- [fermion, half left, momentum=\(k-p\)] b,
  c -- [photon, momentum=\(p\)] d,
};
\end{codeexample}

\begin{codeexample}[]
\feynmandiagram [layered layout, horizontal=a to b] [edges=gluon] {
  {i1, i2} -- a -- [half left] b -- [half left] a,
  b -- {f1, f2},
};
\end{codeexample}

\newpage
\item[Box Diagrams] \hspace*{0pt} \newline
\begin{codeexample}[]
\feynmandiagram [layered layout, horizontal=a to b] {
  % Draw the top and bottom lines
  i1 [particle=\(d\)]
     -- [fermion] a
     -- [photon, edge label=\(W^{-}\)] b
     -- [fermion] f1 [particle=\(\mu^{-}\)],
  i2 [particle=\(\overline s\)]
     -- [anti fermion] c
     -- [photon, edge label'=\(W^{+}\)] d
     -- [anti fermion] f2 [particle=\(\mu^{+}\)],
  % Draw the two internal fermion lines
  { [same layer] a -- [fermion,      edge label'=\(q\)] c },
  { [same layer] b -- [anti fermion, edge label=\(\nu_{\mu}\)] d},
};
\end{codeexample}

\newpage
\item[Meson decay and mixing] \hspace*{0pt} \newline
\begin{codeexample}[]
\begin{tikzpicture}
  \begin{feynman}
    \vertex (a1) {\(\overline b\)};
    \vertex[right=1.5cm of a1] (a2);
    \vertex[right=1cm of a2] (a3);
    \vertex[right=1.5cm of a3] (a4) {\(\overline u\)};

    \vertex[below=2em of a1] (b1) {\(d\)};
    \vertex[below=2em of a4] (b2) {\(d\)};

    %% See section 13.5 of PGF/TikZ manual
    \vertex at ($(a2)!0.5!(a3)!0.5cm!90:(a3)$) (d);
    %% Equivalent way to obtain (d):
    % \vertex at ($(b2)!0.5!(b3) + (0, -0.5cm)$) (d);
    \vertex[above=of a4] (c1) {\(u\)};
    \vertex[above=2em of c1] (c3) {\(\overline d\)};
    \vertex at ($(c1)!0.5!(c3) - (1cm, 0)$) (c2);

    \diagram* {
      (a4) -- [fermion] (a3) -- [fermion] (a2) -- [fermion] (a1),
      (b1) -- [fermion] (b2),
      (c3) -- [fermion, out=180, in=45] (c2) -- [fermion, out=-45, in=180] (c1),
      (a2) -- [boson, quarter left] (d) -- [boson, quarter left] (a3),
      (d) -- [boson, bend left, edge label=\(W^{+}\)] (c2),
    };

    \draw [decoration={brace}, decorate] (b1.south west) -- (a1.north west)
          node [pos=0.5, left] {\(B^{0}\)};
    \draw [decoration={brace}, decorate] (c3.north east) -- (c1.south east)
          node [pos=0.5, right] {\(\pi^{+}\)};
    \draw [decoration={brace}, decorate] (a4.north east) -- (b2.south east)
          node [pos=0.5, right] {\(\pi^{-}\)};
  \end{feynman}
\end{tikzpicture}
\end{codeexample}

\newpage
\begin{codeexample}[]
\begin{tikzpicture}
  \begin{feynman}
    \vertex (a1) {\(\overline b\)};
    \vertex[right=1cm of a1] (a2);
    \vertex[right=1cm of a2] (a3);
    \vertex[right=1cm of a3] (a4) {\(b\)};
    \vertex[right=1cm of a4] (a5);
    \vertex[right=2cm of a5] (a6) {\(u\)};

    \vertex[below=2em of a1] (b1) {\(d\)};
    \vertex[right=1cm of b1] (b2);
    \vertex[right=1cm of b2] (b3);
    \vertex[right=1cm of b3] (b4) {\(\overline d\)};
    \vertex[below=2em of a6] (b5) {\(\overline d\)};

    \vertex[above=of a6] (c1) {\(\overline u\)};
    \vertex[above=2em of c1] (c3) {\(d\)};
    \vertex at ($(c1)!0.5!(c3) - (1cm, 0)$) (c2);

    \diagram* {
      {[edges=fermion]
        (b1) -- (b2) -- (a2) -- (a1),
        (b5) -- (b4) -- (b3) -- (a3) -- (a4) -- (a5) -- (a6),
      },
      (a2) -- [boson, edge label=\(W\)] (a3),
      (b2) -- [boson, edge label'=\(W\)] (b3),

      (c1) -- [fermion, out=180, in=-45] (c2) -- [fermion, out=45, in=180] (c3),
      (a5) -- [boson, bend left, edge label=\(W^{-}\)] (c2),
    };

    \draw [decoration={brace}, decorate] (b1.south west) -- (a1.north west)
          node [pos=0.5, left] {\(B^{0}\)};
    \draw [decoration={brace}, decorate] (c3.north east) -- (c1.south east)
          node [pos=0.5, right] {\(\pi^{-}\)};
    \draw [decoration={brace}, decorate] (a6.north east) -- (b5.south east)
          node [pos=0.5, right] {\(\pi^{+}\)};
  \end{feynman}
\end{tikzpicture}
\end{codeexample}

\newpage
\begin{codeexample}[]
\begin{tikzpicture}
  \begin{feynman}
    \vertex (a1) {\(\overline b\)};
    \vertex[right=2cm of a1] (a2);
    \vertex[right=0.5cm of a2] (a3);
    \vertex[right=0.5cm of a3] (a4);
    \vertex[right=2cm of a4] (a5) {\(\overline s\)};

    \vertex[below=2cm of a1] (b1) {\(d\)};
    \vertex[below=2cm of a5] (b2) {\(d\)};

    \vertex[below=1.5em of a5] (c1) {\(s\)};
    \vertex[above=1.5em of b2] (c3) {\(\overline s\)};
    \vertex at ($(c1)!0.5!(c3) - (1cm, 0)$) (c2);

    \diagram* {
      {[edges=fermion]
        (a5) -- (a4) -- (a3) -- (a2) -- (a1),
      },
      (b1) -- [fermion] (b2),
      (c3) -- [fermion, out=180, in=-60] (c2) -- [fermion, out=60, in=180] (c1),
      (a3) -- [gluon, bend right] (c2),
      (a4) -- [boson, out=90, in=90, looseness=2.0, edge label'=\(W^{+}\)] (a2)
    };

    \draw [decoration={brace}, decorate] (b1.south west) -- (a1.north west)
          node [pos=0.5, left] {\(B^{0}\)};
    \draw [decoration={brace}, decorate] (a5.north east) -- (c1.south east)
          node [pos=0.5, right] {\(\phi\)};
    \draw [decoration={brace}, decorate] (c3.north east) -- (b2.south east)
          node [pos=0.5, right] {\(K^{0}\)};
  \end{feynman}
\end{tikzpicture}
\end{codeexample}

\end{description}

%% Index
%%%%%%%%%%%%%%%%%%%%%%%%%%%%%%%%%%%%%%%%%%%%%%%%%%%%%%%%%%%%%%%%%%%%%%%%%%%%%%%%
\clearpage
\printindex

%% Bibliography (and acknowledgements)
%%%%%%%%%%%%%%%%%%%%%%%%%%%%%%%%%%%%%%%%%%%%%%%%%%%%%%%%%%%%%%%%%%%%%%%%%%%%%%%%
\clearpage
\pagestyle{plain}

\ifarxiv
\acknowledgements
\fi

\printbibliography

\end{document}

%%% Local Variables:
%%% mode: latex
%%% TeX-master: t
%%% End:
